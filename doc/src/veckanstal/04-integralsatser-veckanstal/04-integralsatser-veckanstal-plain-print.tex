%%
%% Automatically generated file from DocOnce source
%% (https://github.com/hplgit/doconce/)
%%
%%


%-------------------- begin preamble ----------------------

\documentclass[%
oneside,                 % oneside: electronic viewing, twoside: printing
final,                   % draft: marks overfull hboxes, figures with paths
10pt]{article}

\listfiles               %  print all files needed to compile this document

\usepackage{relsize,makeidx,color,setspace,amsmath,amsfonts,amssymb}
\usepackage[table]{xcolor}
\usepackage{bm,ltablex,microtype}

\usepackage[pdftex]{graphicx}

\usepackage[T1]{fontenc}
%\usepackage[latin1]{inputenc}
\usepackage{ucs}
\usepackage[utf8x]{inputenc}

\usepackage{lmodern}         % Latin Modern fonts derived from Computer Modern

% Hyperlinks in PDF:
\definecolor{linkcolor}{rgb}{0,0,0.4}
\usepackage{hyperref}
\hypersetup{
    breaklinks=true,
    colorlinks=true,
    linkcolor=linkcolor,
    urlcolor=linkcolor,
    citecolor=black,
    filecolor=black,
    %filecolor=blue,
    pdfmenubar=true,
    pdftoolbar=true,
    bookmarksdepth=3   % Uncomment (and tweak) for PDF bookmarks with more levels than the TOC
    }
%\hyperbaseurl{}   % hyperlinks are relative to this root

\setcounter{tocdepth}{2}  % levels in table of contents

% prevent orhpans and widows
\clubpenalty = 10000
\widowpenalty = 10000

\newenvironment{doconceexercise}{}{}
\newcounter{doconceexercisecounter}


% ------ header in subexercises ------
%\newcommand{\subex}[1]{\paragraph{#1}}
%\newcommand{\subex}[1]{\par\vspace{1.7mm}\noindent{\bf #1}\ \ }
\makeatletter
% 1.5ex is the spacing above the header, 0.5em the spacing after subex title
\newcommand\subex{\@startsection*{paragraph}{4}{\z@}%
                  {1.5ex\@plus1ex \@minus.2ex}%
                  {-0.5em}%
                  {\normalfont\normalsize\bfseries}}
\makeatother


% --- end of standard preamble for documents ---


% insert custom LaTeX commands...

\raggedbottom
\makeindex
\usepackage[totoc]{idxlayout}   % for index in the toc
\usepackage[nottoc]{tocbibind}  % for references/bibliography in the toc

%-------------------- end preamble ----------------------

\begin{document}

% matching end for #ifdef PREAMBLE

\newcommand{\exercisesection}[1]{\subsection*{#1}}

% This file is to be run by preprocess to produce newcommands.tex
% to be included in .tex files.
% There are format-specific tests here for the newcommands (i.e.,
% different definitions of the commands depending on latex or mathjax).

% Newcommands for LaTeX math.
\newcommand{\tp}{\thinspace .}
\renewcommand{\Re}{\bbbr}
\newcommand{\Oof}[1]{\mathcal{O}(#1)}
\newcommand{\Prob}[1]{\hbox{P}(#1)}
\newcommand{\Var}[1]{\hbox{Var}(#1)}
\newcommand{\Cov}[2]{\hbox{Cov}(#1,#2)}
\newcommand{\StDev}[1]{\hbox{StDev}(#1)}

\newcommand{\punkt}{\thinspace .}
\newcommand{\komma}{\thinspace ,}

\newcommand{\vr}{\vec{r}}
\newcommand{\vrp}{\vec{r}\,'}
\newcommand{\erf}{\mathrm{erf}}
\newcommand{\vrho}{\vec{\varrho}}
\newcommand{\vrhop}{\vec{\varrho}\, '}
\newcommand{\sign}{\mathrm{sign}}

\newcommand{\Tr}[1]{\mathrm{Tr}[#1]}
\newcommand{\e}{\varepsilon}
\newcommand{\g}{\gamma}

\newcommand{\half}{\frac{1}{2}}
\newcommand{\vnabla}{\vec{\nabla}}


% Use footnotesize in subscripts
\newcommand{\subsc}[2]{#1_{\mbox{\footnotesize #2}}}




% ------------------- main content ----------------------



% ----------------- title -------------------------

\thispagestyle{empty}

\begin{center}
{\LARGE\bf
\begin{spacing}{1.25}
FFM234, Klassisk fysik och vektorfält - Veckans tal
\end{spacing}
}
\end{center}

% ----------------- author(s) -------------------------

\begin{center}
{\bf \href{{http://fy.chalmers.se/subatom/tsp/}}{Christian Forssén}, Institutionen för  fysik, Chalmers${}^{}$} \\ [0mm]
\end{center}

\begin{center}
% List of all institutions:
\end{center}
    
% ----------------- end author(s) -------------------------

% --- begin date ---
\begin{center}
Aug 10, 2019
\end{center}
% --- end date ---

\vspace{1cm}


% --- begin exercise ---
\begin{doconceexercise}
\refstepcounter{doconceexercisecounter}

\subsection*{Kurvintegral längs komplicerad ellips}

Beräkna integralen
\begin{equation}
  \oint_\Gamma \vec{F} \cdot \mbox{d}\vec{r},
\end{equation}
där 
\begin{equation}
  \vec{F} = \left[x^2-a\left(y+z\right)\right]\hat{x} + \left(y^2-az\right)
\hat{y} + \left[z^2-a\left(x+y\right)\right] \hat{z},
\end{equation}
och $\Gamma$ är den kurva som utgör skärningen mellan cylindern
\begin{equation}
 \left(x-a\right)^2 +y^2 = a^2,\quad z \ge 0,
\end{equation}
och sfären
\begin{equation}
  x^2 + y^2 + z^2 = R^2, \quad R> 2a,
\end{equation}
där $a$ är en konstant med dimensionen längd.


% --- begin answer of exercise ---
\paragraph{Answer.}
$\pi a^3$.

% --- end answer of exercise ---


% --- begin solution of exercise ---
\paragraph{Solution.}
Vi kan först konstatera att skärningen mellan cylinder och sfär är en ellips vars exakta form är något komplicerad att fastställa.  Eftersom kurvan $\Gamma$ är en sluten kurva är det lockande att använda Stokes sats, så vi beräknar rotationen
\begin{align}
  \vec{\nabla} \times \vec{F} &= 
  \left|\begin{array}{ccc}
\hat{x} & \hat{y} & \hat{z} \\
\frac{\partial}{\partial x} & \frac{\partial}{\partial y} & 
\frac{\partial}{\partial z} \\
x^2-a\left(y+z\right) & y^2-az & z^2-a\left(x+y\right) \\
\end{array} 
\right| \nonumber \\
&= \left(-a+a\right) \hat{x} + \left(-a+a\right) \hat{y} + a\hat{z}
= a\hat{z}.
\end{align}
Alltså är rotationen av $\vec{F}$ en rent vertikal vektor.  

Vi kan nu använda Stokes sats
\begin{equation}
  \oint_\Gamma \vec{F} \cdot \mbox{d}\vec{r} = \int_S \vec{\nabla} \times \vec{F} \cdot
\mbox{d}\vec{S}.
\end{equation}
Lägg märke till att ytan skall orienteras så att den följer högerhandsregeln.  Detta betyder att om vi följer kurvan $\Gamma$ moturs så skall normalen $\hat{n}$ till $S$ peka uppåt.  
\begin{equation}
  \int_S \vec{\nabla} \times \vec{F} \cdot \mbox{d}\vec{S} = \int_S a \hat{z} \cdot 
\hat{n} \mbox{d}S = a \int_S \hat{z} \cdot \hat{n} \mbox{d}S.
\end{equation}
Skalärprodukten i den sista integralen betyder att vi projicerar ner arean $S$ på ett plan vinkelrät mot $\hat{z}$, det vill säga på $xy$-planet.  I detta planet är skärningen cylinderns tvärsnittsyta, en cirkel med radien $a$, och integralen blir cirkelarean $\pi a^2$. Alltså blir integralen till slut
\begin{equation}
  \oint_\Gamma \vec{F} \cdot \mbox{d}\vec{r} = a \pi a^2 = \pi a^3.
\end{equation}

% --- end solution of exercise ---

\end{doconceexercise}
% --- end exercise ---


% ------------------- end of main content ---------------

\end{document}

