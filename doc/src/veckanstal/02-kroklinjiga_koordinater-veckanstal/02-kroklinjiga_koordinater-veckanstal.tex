%%
%% Automatically generated file from DocOnce source
%% (https://github.com/hplgit/doconce/)
%%
%%


%-------------------- begin preamble ----------------------

\documentclass[%
oneside,                 % oneside: electronic viewing, twoside: printing
final,                   % draft: marks overfull hboxes, figures with paths
10pt]{article}

\listfiles               %  print all files needed to compile this document

\usepackage{relsize,makeidx,color,setspace,amsmath,amsfonts,amssymb}
\usepackage[table]{xcolor}
\usepackage{bm,ltablex,microtype}

\usepackage[pdftex]{graphicx}

\usepackage[T1]{fontenc}
%\usepackage[latin1]{inputenc}
\usepackage{ucs}
\usepackage[utf8x]{inputenc}

\usepackage{lmodern}         % Latin Modern fonts derived from Computer Modern

% Hyperlinks in PDF:
\definecolor{linkcolor}{rgb}{0,0,0.4}
\usepackage{hyperref}
\hypersetup{
    breaklinks=true,
    colorlinks=true,
    linkcolor=linkcolor,
    urlcolor=linkcolor,
    citecolor=black,
    filecolor=black,
    %filecolor=blue,
    pdfmenubar=true,
    pdftoolbar=true,
    bookmarksdepth=3   % Uncomment (and tweak) for PDF bookmarks with more levels than the TOC
    }
%\hyperbaseurl{}   % hyperlinks are relative to this root

\setcounter{tocdepth}{2}  % levels in table of contents

% prevent orhpans and widows
\clubpenalty = 10000
\widowpenalty = 10000

\newenvironment{doconceexercise}{}{}
\newcounter{doconceexercisecounter}


% ------ header in subexercises ------
%\newcommand{\subex}[1]{\paragraph{#1}}
%\newcommand{\subex}[1]{\par\vspace{1.7mm}\noindent{\bf #1}\ \ }
\makeatletter
% 1.5ex is the spacing above the header, 0.5em the spacing after subex title
\newcommand\subex{\@startsection*{paragraph}{4}{\z@}%
                  {1.5ex\@plus1ex \@minus.2ex}%
                  {-0.5em}%
                  {\normalfont\normalsize\bfseries}}
\makeatother


% --- end of standard preamble for documents ---


% insert custom LaTeX commands...

\raggedbottom
\makeindex
\usepackage[totoc]{idxlayout}   % for index in the toc
\usepackage[nottoc]{tocbibind}  % for references/bibliography in the toc

%-------------------- end preamble ----------------------

\begin{document}

% matching end for #ifdef PREAMBLE

\newcommand{\exercisesection}[1]{\subsection*{#1}}

% This file is to be run by preprocess to produce newcommands.tex
% to be included in .tex files.
% There are format-specific tests here for the newcommands (i.e.,
% different definitions of the commands depending on latex or mathjax).

% Newcommands for LaTeX math.
\newcommand{\tp}{\thinspace .}
\renewcommand{\Re}{\bbbr}
\newcommand{\Oof}[1]{\mathcal{O}(#1)}
\newcommand{\Prob}[1]{\hbox{P}(#1)}
\newcommand{\Var}[1]{\hbox{Var}(#1)}
\newcommand{\Cov}[2]{\hbox{Cov}(#1,#2)}
\newcommand{\StDev}[1]{\hbox{StDev}(#1)}

\newcommand{\punkt}{\thinspace .}
\newcommand{\komma}{\thinspace ,}

\newcommand{\vr}{\vec{r}}
\newcommand{\vrp}{\vec{r}\,'}
\newcommand{\erf}{\mathrm{erf}}
\newcommand{\vrho}{\vec{\varrho}}
\newcommand{\vrhop}{\vec{\varrho}\, '}
\newcommand{\sign}{\mathrm{sign}}

\newcommand{\Tr}[1]{\mathrm{Tr}[#1]}
\newcommand{\e}{\varepsilon}
\newcommand{\g}{\gamma}

\newcommand{\half}{\frac{1}{2}}
\newcommand{\vnabla}{\vec{\nabla}}


% Use footnotesize in subscripts
\newcommand{\subsc}[2]{#1_{\mbox{\footnotesize #2}}}




% ------------------- main content ----------------------



% ----------------- title -------------------------

\thispagestyle{empty}

\begin{center}
{\LARGE\bf
\begin{spacing}{1.25}
FFM234, Klassisk fysik och vektorfält - Veckans tal
\end{spacing}
}
\end{center}

% ----------------- author(s) -------------------------

\begin{center}
{\bf \href{{http://fy.chalmers.se/subatom/tsp/}}{Christian Forssén} och Kevin Marc Seja, Institutionen för  fysik, Chalmers${}^{}$} \\ [0mm]
\end{center}

\begin{center}
% List of all institutions:
\end{center}
    
% ----------------- end author(s) -------------------------

% --- begin date ---
\begin{center}
Sep 26, 2020
\end{center}
% --- end date ---

\vspace{1cm}


% --- begin exercise ---
\begin{doconceexercise}
\refstepcounter{doconceexercisecounter}

\subsection*{Uppgift 2.4.8}

Betrakta vektorfältet
$$
\vec E(\vec{r}) = \frac{m}{4\pi r^3} (2\cos \theta \hat{r} + \sin\theta \hat{\theta}),
$$
där $m$ är en konstant. (Detta är fältet från en elektrisk dipol.)

Bestäm ekvationen för den fältlinje till $\vec E(\vec{r})$ som går genom punkten $(r, \theta, \varphi) = (2, \pi /4, \pi /6)$.

% --- begin hint in exercise ---

\paragraph{Hint.}
Fältlinjer är de kurvor som följer ett vektorfält på så sätt att de i varje punkt har vektorfältet som sin tangentvektor. Fältlinjer kan parametriseras $\vec{r} = \vec{r}(\tau)$ och differentialekvationerna för att bestämma dem är
$$
\frac{\mbox{d}\vec{r}}{\mbox{d}\tau} = C \vec{E},
$$
där $C$ är en godtycklig konstant vilken ju inte påverkar tangentriktningen.

Med cartesiska koordinater gäller ju att förskjutningsvektorn $\mbox{d}\vec{r}$ kan skrivas $\mbox{d}\vec{r} = \hat{x} \mbox{d}x + \hat{y} \mbox{d}y + \hat{z} \mbox{d}z$ och vektorekvationen ovan ger tre differentialekvationer (en för varje riktning $\hat{x}$, $\hat{y}$, $\hat{z}$):
$$
\left\{
\begin{array}{ll}
x: &
\frac{\mbox{d}x}{\mbox{d}\tau} = C E_x \\
y: &
\frac{\mbox{d}y}{\mbox{d}\tau} = C E_y \\
z: &
\frac{\mbox{d}z}{\mbox{d}\tau} = C E_z.
\end{array}
\right.
$$
Men om fältet är mycket enklare att uttrycka i kroklinjiga koordinater är det fördelaktigt att teckna differentialekvationerna i dessa riktningarna istället. Men då får man komma ihåg att förskjutningsvektorn blir
$$
\mbox{d}\vec{r} = \sum_{i=1}^3 h_i \hat{u}_i \mbox{d}u_i,
$$
där $h_i$ är koordinatsystemets skalfaktorer.

% --- end hint in exercise ---


% --- begin answer of exercise ---
\paragraph{Answer.}
$r = 4 \sin^2 \theta$, $\varphi = \pi/6$.

% --- end answer of exercise ---


% --- begin solution of exercise ---
\paragraph{Solution.}
I sfäriska koordinater är de tre skalfaktorerna
$$
h_r = 1, h_\theta = r, h_\varphi = r \sin \theta.
$$
Förskjutningsvektorn är
$$
\mbox{d}\vec{r} = h_r \hat{r}~\mbox{d}r + h_\theta \hat{\theta} ~\mbox{d}\theta + h_\varphi \hat{\varphi} ~ \mbox{d}\varphi.
$$
Vi får de tre separata differentialekvationerna genom att ta skalärprodukten av differentialekvationen
$$ \frac{\mbox{d}\vec{r} }{ \mbox{d}\tau} = C \vec{E},$$
med de tre basvektorerna $\hat{r}, \hat{\theta}, \hat{\varphi}$. Vi väljer $ C = 4\pi/m$. Detta leder till
$$ \underbrace{h_r}_{=1} \frac{\mbox{d}r}{\mbox{d}\tau} = \frac{2 \cos \theta}{r^3}  = E_r$$
$$  \underbrace{h_\theta}_{= r } \frac{\mbox{d}\theta}{\mbox{d}\tau} = \frac{\sin \theta }{r^3} = E_\theta$$
$$ \underbrace{h_\varphi}_{=r \sin \theta} \frac{\mbox{d} \varphi}{\mbox{d}\tau} = 0 = E_\phi$$
Ekvationen för $\varphi$ leder direkt till $\varphi = \mbox{const.} = \pi/6 = \varphi_0 $. De andra två ekvationer kan skrivas som
$$ \frac{\mbox{d}r}{\mbox{d}\tau} = \frac{2 \cos \theta}{ r^3}, $$
$$ \frac{\mbox{d}\theta}{\mbox{d}\tau} = \frac{\sin \theta}{ r^4}.$$
Vi kombinerar de två ekvationer för att få
$$ \frac{\mbox{d}\theta}{ \mbox{d} r} = \frac{1}{2r} \frac{\sin \theta}{\cos \theta} = \frac{1}{2r} \tan\theta.$$
Detta leder då till 
$$ \frac{\mbox{d}\theta}{\tan \theta} = \frac{\mbox{d}r}{2r}.$$
Vi integrerar både sidor med startpunkt i den givna punkten $(r_0 = 2, \theta_0 = \pi/4, \varphi_0=\pi/6)$ som startvärde och får
$$ \ln \left| \frac{\sin\theta}{\sin \theta_0} \right| = \frac{1}{2} \ln \left| \frac{r}{r_0} \right|$$
[Anm.: Vänstersidans integral finns tabulerad.] 
\\Man använder nu räkneregler för logaritmer och resultatet blir 
$$ \sin^2 \theta = \sin^2 \theta_0 \frac{1}{r_0} r.$$ För det givna startvärdet får man $\sin^2 \theta_0 = \frac{1}{2}$, $r_0 = 2$, så att fältlinjerna beskrivs av
$$ \sin^2 \theta = \frac{1}{4} r \Leftrightarrow r = 4 \sin^2 \theta$$ och $\varphi = \pi/6 = \mbox{const.}$ hade vi visat tidigare.

% --- end solution of exercise ---

% Closing remarks for this Exercise

\paragraph{Remarks.}
Uppgiften illustrerar hur man ställer upp differentialekvationerna för fältlinjer i fallet då vektorfältet enklast beskrivs i ett kroklinjigt koordinatsystem. Den illustrerar också hur en specifik fältlinje kan identifieras från ett randvillkor.


\end{doconceexercise}
% --- end exercise ---


% ------------------- end of main content ---------------

\end{document}

