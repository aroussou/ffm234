%%
%% Automatically generated file from DocOnce source
%% (https://github.com/hplgit/doconce/)
%%
%%


%-------------------- begin preamble ----------------------

\documentclass[%
oneside,                 % oneside: electronic viewing, twoside: printing
final,                   % draft: marks overfull hboxes, figures with paths
10pt]{article}

\listfiles               %  print all files needed to compile this document

\usepackage{relsize,makeidx,color,setspace,amsmath,amsfonts,amssymb}
\usepackage[table]{xcolor}
\usepackage{bm,ltablex,microtype}

\usepackage[pdftex]{graphicx}

\usepackage[T1]{fontenc}
%\usepackage[latin1]{inputenc}
\usepackage{ucs}
\usepackage[utf8x]{inputenc}

\usepackage{lmodern}         % Latin Modern fonts derived from Computer Modern

% Hyperlinks in PDF:
\definecolor{linkcolor}{rgb}{0,0,0.4}
\usepackage{hyperref}
\hypersetup{
    breaklinks=true,
    colorlinks=true,
    linkcolor=linkcolor,
    urlcolor=linkcolor,
    citecolor=black,
    filecolor=black,
    %filecolor=blue,
    pdfmenubar=true,
    pdftoolbar=true,
    bookmarksdepth=3   % Uncomment (and tweak) for PDF bookmarks with more levels than the TOC
    }
%\hyperbaseurl{}   % hyperlinks are relative to this root

\setcounter{tocdepth}{2}  % levels in table of contents

% prevent orhpans and widows
\clubpenalty = 10000
\widowpenalty = 10000

\newenvironment{doconceexercise}{}{}
\newcounter{doconceexercisecounter}


% ------ header in subexercises ------
%\newcommand{\subex}[1]{\paragraph{#1}}
%\newcommand{\subex}[1]{\par\vspace{1.7mm}\noindent{\bf #1}\ \ }
\makeatletter
% 1.5ex is the spacing above the header, 0.5em the spacing after subex title
\newcommand\subex{\@startsection*{paragraph}{4}{\z@}%
                  {1.5ex\@plus1ex \@minus.2ex}%
                  {-0.5em}%
                  {\normalfont\normalsize\bfseries}}
\makeatother


% --- end of standard preamble for documents ---


% insert custom LaTeX commands...

\raggedbottom
\makeindex
\usepackage[totoc]{idxlayout}   % for index in the toc
\usepackage[nottoc]{tocbibind}  % for references/bibliography in the toc

%-------------------- end preamble ----------------------

\begin{document}

% matching end for #ifdef PREAMBLE

\newcommand{\exercisesection}[1]{\subsection*{#1}}

% This file is to be run by preprocess to produce newcommands.tex
% to be included in .tex files.
% There are format-specific tests here for the newcommands (i.e.,
% different definitions of the commands depending on latex or mathjax).

% Newcommands for LaTeX math.
\newcommand{\tp}{\thinspace .}
\renewcommand{\Re}{\bbbr}
\newcommand{\Oof}[1]{\mathcal{O}(#1)}
\newcommand{\Prob}[1]{\hbox{P}(#1)}
\newcommand{\Var}[1]{\hbox{Var}(#1)}
\newcommand{\Cov}[2]{\hbox{Cov}(#1,#2)}
\newcommand{\StDev}[1]{\hbox{StDev}(#1)}

\newcommand{\punkt}{\thinspace .}
\newcommand{\komma}{\thinspace ,}

\newcommand{\vr}{\vec{r}}
\newcommand{\vrp}{\vec{r}\,'}
\newcommand{\erf}{\mathrm{erf}}
\newcommand{\vrho}{\vec{\varrho}}
\newcommand{\vrhop}{\vec{\varrho}\, '}
\newcommand{\sign}{\mathrm{sign}}

\newcommand{\Tr}[1]{\mathrm{Tr}[#1]}
\newcommand{\e}{\varepsilon}
\newcommand{\g}{\gamma}

\newcommand{\half}{\frac{1}{2}}
\newcommand{\vnabla}{\vec{\nabla}}


% Use footnotesize in subscripts
\newcommand{\subsc}[2]{#1_{\mbox{\footnotesize #2}}}




% ------------------- main content ----------------------



% ----------------- title -------------------------

\thispagestyle{empty}

\begin{center}
{\LARGE\bf
\begin{spacing}{1.25}
FFM234, Klassisk fysik och vektorfält - Veckans tal
\end{spacing}
}
\end{center}

% ----------------- author(s) -------------------------

\begin{center}
{\bf Christin Rhen och Christian Forssén, Institutionen för  fysik, Chalmers${}^{}$} \\ [0mm]
\end{center}

\begin{center}
% List of all institutions:
\end{center}
    
% ----------------- end author(s) -------------------------

% --- begin date ---
\begin{center}
Aug 10, 2019
\end{center}
% --- end date ---

\vspace{1cm}


% --- begin exercise ---
\begin{doconceexercise}
\refstepcounter{doconceexercisecounter}

\subsection*{Uppgift 2.4.14}
\label{problem:2.14}

Ett kroklinjigt koordinatsystem $u,v,w$ ges av sambanden 
\begin{align}
    u&=r(1-\cos\theta),\nonumber\\
    v&=r(1+\cos\theta),\label{uvw}\\
    w&=\varphi,\nonumber
\end{align}
där $r,\theta,\varphi$ är sfäriska koordinater. Visa att systemet är ortogonalt och beräkna dess skalfaktorer. Hur ser gradientoperatorn $\nabla$ och ortvektorn $\vec r$ ut i $u,v,w$-systemet?


% --- begin answer of exercise ---
\paragraph{Answer.}
Skalfaktorer:
\begin{align*}
h_u  &=  \frac{1}{2} \sqrt{\frac{u+v}{u}} \\
h_v  &=  \frac{1}{2} \sqrt{\frac{u+v}{v}} \\
h_w  &=  \sqrt{uv}
\end{align*}
Gradient: 
\begin{equation*}
\nabla = \frac{2}{\sqrt{u+v}}\left(\sqrt{u} \hat u
\frac{\partial}{\partial u} + \sqrt{v} \hat v
\frac{\partial}{\partial v}\right) + \frac{1}{\sqrt{uv}} \hat w
\frac{\partial}{\partial w}
\end{equation*}
Ortsvektor: 
\begin{equation*}
\vec{r} = \frac{\sqrt{u+v}}{2}(\sqrt{u} \hat u + \sqrt{v} \hat v)
\end{equation*}

% --- end answer of exercise ---


% --- begin solution of exercise ---
\paragraph{Solution.}
\paragraph{Enhetsvektorer}
Enhetsvektorer ges av (långt upp på sida 12 i kurskompendiet)
\begin{equation}
    \vec e_i=h_i\nabla u_i. 
    \label{eq:ehat}
\end{equation}
Här är $u,v,w$ en funktion av sfäriska koordinater, så vi använder den sfäriska gradienten (kurskompendium ekvation 2.14):
\begin{align}
    \hat u\propto\nabla u&= \left(\hat r\frac{\partial}{\partial r}+\hat\theta\frac 1r\frac{\partial}{\partial \theta}+\hat\varphi\frac1{r\sin\theta}\frac{\partial}{\partial \varphi}\right)u\nonumber\\
    &=(1-\cos\theta)\hat r+\sin\theta\hat\theta,\\
    \hat v\propto\nabla v&=(1+\cos\theta)\hat r-\sin\theta\hat\theta,\\
    \hat w\propto\nabla w&=\hat\varphi\frac1{r\sin\theta}.
\end{align}

\paragraph{Ortogonalitet}
Det är lättast att kontrollera om ett nytt system är ortogonalt om vi har dess basvektorer uttryckta i ett annat, mer välkänt, system. Därför låter vi här $\hat u, \hat v,\hat w$ fortsätta vara en funktion av sfäriska koordinater och basvektorer, och räknar ut skalärprodukterna mellan dem. Vi ser direkt att $\hat u\cdot \hat w=\hat v\cdot\hat w=0$. För $\hat u$ och $\hat v$:
\begin{align}
    \hat u\cdot\hat v&\propto(1-\cos\theta)(1+\cos\theta)-\sin^2\theta\nonumber\\
    &=1-\cos^2\theta-\sin^2\theta=0.
\end{align}
Alltså är $u,v,w$-systemet ortogonalt.

\paragraph{Skalfaktorer och enhetsvektorer}
Från ekvation (\ref{eq:ehat}) inser vi att skalfaktorerna $h_i$ är inversen av $|\nabla u_i|$. För enkelhetens skull fortsätter vi räkna i termer av sfäriska koordinater, och översätter till $u,v,w$ i slutet. Notera att $u+v=2r$. Vi får
\begin{align}
    h_u&=\left((1-\cos\theta)^2+\sin^2\theta\right)^{-1/2}=\left(2-2\cos\theta\right)^{-1/2}\nonumber\\
    &=\sqrt{\frac{u+v}{4u}},\\
    h_v&=\left((1+\cos\theta)^2+\sin^2\theta\right)^{-1/2}=\left(2+2\cos\theta\right)^{-1/2}\nonumber\\
    &=\sqrt{\frac{u+v}{4v}},\\
    h_w&=\sqrt{r^2\sin^2\theta}=\sqrt{r^2(1-\cos^2\theta)}=\sqrt{uv}. 
\end{align}

Kombinerar vi skalfaktorerna (i sfäriska koordinater) och de onormerade enhetvektorerna uträknade ovan så får vi 
\begin{align}
    \hat u&= \frac{1-\cos\theta}{\sqrt{2-2\cos\theta}}\hat r+\frac{\sin\theta}{\sqrt{2-2\cos\theta}}\hat\theta,\\
    \hat v&= \frac{1+\cos\theta}{\sqrt{2+2\cos\theta}}\hat r-\frac{\sin\theta}{\sqrt{2+2\cos\theta}}\hat\theta,\\
    \hat w&=\hat\varphi.
    \label{eq:enhetsvektorer}
\end{align}

\paragraph{Gradientoperator}
För ortogonala koordinatsystem gäller att (kurskompendium ekvation 2.13)
\begin{equation}
    \nabla\phi=\sum_i\vec e_i\frac 1{h_i}\frac{\partial\phi}{\partial u_i},
\end{equation}
där $\phi$ är ett godtyckligt skalärfält. I vårt $u,v,w$-system blir detta
\begin{equation}
    \nabla\phi=\hat u\frac 1{h_u}\frac{\partial\phi}{\partial u}+\hat v\frac 1{h_v}\frac{\partial\phi}{\partial v}+\hat w\frac 1{h_w}\frac{\partial\phi}{\partial w}.
\end{equation}
Sätter vi in skalfaktorerna vi beräknat ovan är det lätt att identifiera gradientoperatorn
\begin{equation}
    \nabla = \hat u\sqrt{\frac{4u}{u+v}}\frac{\partial}{\partial u}+\hat v\sqrt{\frac{4v}{u+v}}\frac{\partial}{\partial v}+\hat w\frac 1{\sqrt{uv}}\frac{\partial}{\partial w}.
\end{equation}
Notera att den inversa skalfaktorn alltid kommer före partialderivatan! Skriver man dem i fel ordning ska plötsligt även skalfaktorn deriveras, och det blir fel.

\paragraph{Ortsvektor}
I sfäriska koordinater är ortsvektorn 
\begin{equation}
    \vec r=r\hat r.
\end{equation}
Vi har redan noterat att $r=(u+v)/2$; vi behöver nu också uttrycka $\hat r$ som en funktion av $\hat u,\hat v, \hat w$. 

En smart linjärkombination av enhetsvektorerna (\ref{eq:enhetsvektorer}):
\begin{equation}
    \sqrt{2-2\cos\theta}\hat u+\sqrt{2+2\cos\theta}\hat v=2\hat r,
\end{equation}
som översatt till nya koordinater betyder att
\begin{equation}
    \hat r=\frac12\left(\frac1{h_u}\hat u+\frac1{h_v}\hat v\right)=\sqrt{\frac u{u+v}}\hat u+\sqrt{\frac v{u+v}}\hat v.
\end{equation}
Vi får alltså ortsvektorn
\begin{align}
    \vec r&=r\hat r\nonumber\\
    &=\frac{u+v}2\left(\sqrt{\frac u{u+v}}\hat u+\sqrt{\frac v{u+v}}\hat v\right)\nonumber\\
    &=\tfrac12\sqrt{u+v} \left( \sqrt u\hat u + \sqrt v\hat v \right).
\end{align}

% --- end solution of exercise ---

\end{doconceexercise}
% --- end exercise ---


% ------------------- end of main content ---------------

\end{document}

