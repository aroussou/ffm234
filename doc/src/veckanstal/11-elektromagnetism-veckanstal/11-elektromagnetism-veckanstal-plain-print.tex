%%
%% Automatically generated file from DocOnce source
%% (https://github.com/hplgit/doconce/)
%%
%%


%-------------------- begin preamble ----------------------

\documentclass[%
oneside,                 % oneside: electronic viewing, twoside: printing
final,                   % draft: marks overfull hboxes, figures with paths
10pt]{article}

\listfiles               %  print all files needed to compile this document

\usepackage{relsize,makeidx,color,setspace,amsmath,amsfonts,amssymb}
\usepackage[table]{xcolor}
\usepackage{bm,ltablex,microtype}

\usepackage[pdftex]{graphicx}

\usepackage[T1]{fontenc}
%\usepackage[latin1]{inputenc}
\usepackage{ucs}
\usepackage[utf8x]{inputenc}

\usepackage{lmodern}         % Latin Modern fonts derived from Computer Modern

% Hyperlinks in PDF:
\definecolor{linkcolor}{rgb}{0,0,0.4}
\usepackage{hyperref}
\hypersetup{
    breaklinks=true,
    colorlinks=true,
    linkcolor=linkcolor,
    urlcolor=linkcolor,
    citecolor=black,
    filecolor=black,
    %filecolor=blue,
    pdfmenubar=true,
    pdftoolbar=true,
    bookmarksdepth=3   % Uncomment (and tweak) for PDF bookmarks with more levels than the TOC
    }
%\hyperbaseurl{}   % hyperlinks are relative to this root

\setcounter{tocdepth}{2}  % levels in table of contents

% prevent orhpans and widows
\clubpenalty = 10000
\widowpenalty = 10000

\newenvironment{doconceexercise}{}{}
\newcounter{doconceexercisecounter}


% ------ header in subexercises ------
%\newcommand{\subex}[1]{\paragraph{#1}}
%\newcommand{\subex}[1]{\par\vspace{1.7mm}\noindent{\bf #1}\ \ }
\makeatletter
% 1.5ex is the spacing above the header, 0.5em the spacing after subex title
\newcommand\subex{\@startsection*{paragraph}{4}{\z@}%
                  {1.5ex\@plus1ex \@minus.2ex}%
                  {-0.5em}%
                  {\normalfont\normalsize\bfseries}}
\makeatother


% --- end of standard preamble for documents ---


% insert custom LaTeX commands...

\raggedbottom
\makeindex
\usepackage[totoc]{idxlayout}   % for index in the toc
\usepackage[nottoc]{tocbibind}  % for references/bibliography in the toc

%-------------------- end preamble ----------------------

\begin{document}

% matching end for #ifdef PREAMBLE

\newcommand{\exercisesection}[1]{\subsection*{#1}}


% ------------------- main content ----------------------



% ----------------- title -------------------------

\thispagestyle{empty}

\begin{center}
{\LARGE\bf
\begin{spacing}{1.25}
FFM234, Klassisk fysik och vektorfält - Veckans tal
\end{spacing}
}
\end{center}

% ----------------- author(s) -------------------------

\begin{center}
{\bf \href{{http://fy.chalmers.se/subatom/nt/}}{Christian Forssén}, Institutionen för fysik, Chalmers${}^{}$} \\ [0mm]
\end{center}

\begin{center}
% List of all institutions:
\end{center}
    
% ----------------- end author(s) -------------------------

% --- begin date ---
\begin{center}
Aug 10, 2019
\end{center}
% --- end date ---

\vspace{1cm}


% --- begin exercise ---
\begin{doconceexercise}
\refstepcounter{doconceexercisecounter}

\subsection*{Uppgift 11.8.8}

Bestäm $\vec{E}$-fältet överallt i rummet från en sfäriskt symmetrisk laddningsfördelning bestående av två delar, nämligen en rymdladdning $\rho(r) = \rho_0$ för $r< a/2$ och $\rho = 0$ för $r > a/2$ och en ytladdning $\sigma = -\rho_0 a/24$ på sfären $r = a$.

% --- begin hint in exercise ---

\paragraph{Hint.}
\begin{itemize}
\item Utnyttja den sfäriska symmetrin i problemet som gör att $\vec{E} = E_r \hat{r}$.

\item Använd Maxwells första ekvation som säger att $\nabla \cdot \vec{E} = \rho / \epsilon_0$, där $\rho$ är laddningstätheten.

\item En volymsintegral av laddningstätheten (t.ex. över en sfär med radien $r$) ger den inneslutna laddningen.

\item Motsvarande volymsintegral över VL av Maxwells första ekv (dvs över divergensen av $\vec{E}$-fältet) kan skrivas om med Gauss sats.

\item En ytladdning har enheten laddning/area. Dvs den totala laddningen på det yttre skalet fås genom att integrera över ytan.
\end{itemize}

\noindent
% --- end hint in exercise ---


% --- begin answer of exercise ---
\paragraph{Answer.}
\begin{align}
E\left(r\right) &= \frac{\rho_0}{3 \epsilon_0} r \quad & \mathrm{för~} r < \frac{a}{2} \\
E\left(r\right) &=\frac{\rho_0}{24\epsilon_0} \frac{a^3}{r^2} \quad & \mathrm{för~} \frac{a}{2} < r < a \\
E\left(r\right) &= 0 \quad & \mathrm{för~} r > a
\end{align}

% --- end answer of exercise ---


% --- begin solution of exercise ---
\paragraph{Solution.}
Att göra

% --- end solution of exercise ---

% Closing remarks for this Exercise

\paragraph{Remarks.}
Uppgiften illustrerar användandet av Maxwells ekvationer och hur vi kan utnyttja en integralsats.


\end{doconceexercise}
% --- end exercise ---


% ------------------- end of main content ---------------

\end{document}

