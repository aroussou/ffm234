%%
%% Automatically generated file from DocOnce source
%% (https://github.com/hplgit/doconce/)
%%
%%
% #ifdef PTEX2TEX_EXPLANATION
%%
%% The file follows the ptex2tex extended LaTeX format, see
%% ptex2tex: http://code.google.com/p/ptex2tex/
%%
%% Run
%%      ptex2tex myfile
%% or
%%      doconce ptex2tex myfile
%%
%% to turn myfile.p.tex into an ordinary LaTeX file myfile.tex.
%% (The ptex2tex program: http://code.google.com/p/ptex2tex)
%% Many preprocess options can be added to ptex2tex or doconce ptex2tex
%%
%%      ptex2tex -DMINTED myfile
%%      doconce ptex2tex myfile envir=minted
%%
%% ptex2tex will typeset code environments according to a global or local
%% .ptex2tex.cfg configure file. doconce ptex2tex will typeset code
%% according to options on the command line (just type doconce ptex2tex to
%% see examples). If doconce ptex2tex has envir=minted, it enables the
%% minted style without needing -DMINTED.
% #endif

% #define PREAMBLE

% #ifdef PREAMBLE
%-------------------- begin preamble ----------------------

\documentclass[%
oneside,                 % oneside: electronic viewing, twoside: printing
final,                   % draft: marks overfull hboxes, figures with paths
10pt]{article}

\listfiles               %  print all files needed to compile this document

\usepackage{relsize,makeidx,color,setspace,amsmath,amsfonts,amssymb}
\usepackage[table]{xcolor}
\usepackage{bm,ltablex,microtype}

\usepackage[pdftex]{graphicx}

\usepackage[T1]{fontenc}
%\usepackage[latin1]{inputenc}
\usepackage{ucs}
\usepackage[utf8x]{inputenc}

\usepackage{lmodern}         % Latin Modern fonts derived from Computer Modern

% Hyperlinks in PDF:
\definecolor{linkcolor}{rgb}{0,0,0.4}
\usepackage{hyperref}
\hypersetup{
    breaklinks=true,
    colorlinks=true,
    linkcolor=linkcolor,
    urlcolor=linkcolor,
    citecolor=black,
    filecolor=black,
    %filecolor=blue,
    pdfmenubar=true,
    pdftoolbar=true,
    bookmarksdepth=3   % Uncomment (and tweak) for PDF bookmarks with more levels than the TOC
    }
%\hyperbaseurl{}   % hyperlinks are relative to this root

\setcounter{tocdepth}{2}  % levels in table of contents

% prevent orhpans and widows
\clubpenalty = 10000
\widowpenalty = 10000

\newenvironment{doconceexercise}{}{}
\newcounter{doconceexercisecounter}


% ------ header in subexercises ------
%\newcommand{\subex}[1]{\paragraph{#1}}
%\newcommand{\subex}[1]{\par\vspace{1.7mm}\noindent{\bf #1}\ \ }
\makeatletter
% 1.5ex is the spacing above the header, 0.5em the spacing after subex title
\newcommand\subex{\@startsection{paragraph}{4}{\z@}%
                  {1.5ex\@plus1ex \@minus.2ex}%
                  {-0.5em}%
                  {\normalfont\normalsize\bfseries}}
\makeatother


% --- end of standard preamble for documents ---


% insert custom LaTeX commands...

\raggedbottom
\makeindex
\usepackage[totoc]{idxlayout}   % for index in the toc
\usepackage[nottoc]{tocbibind}  % for references/bibliography in the toc

%-------------------- end preamble ----------------------

\begin{document}

% matching end for #ifdef PREAMBLE
% #endif

\newcommand{\exercisesection}[1]{\subsection*{#1}}

% This file is to be run by preprocess to produce newcommands.tex
% to be included in .tex files.
% There are format-specific tests here for the newcommands (i.e.,
% different definitions of the commands depending on latex or mathjax).

% Newcommands for LaTeX math.
\newcommand{\tp}{\thinspace .}
\renewcommand{\Re}{\bbbr}
\newcommand{\Oof}[1]{\mathcal{O}(#1)}
\newcommand{\Prob}[1]{\hbox{P}(#1)}
\newcommand{\Var}[1]{\hbox{Var}(#1)}
\newcommand{\Cov}[2]{\hbox{Cov}(#1,#2)}
\newcommand{\StDev}[1]{\hbox{StDev}(#1)}

\newcommand{\punkt}{\thinspace .}
\newcommand{\komma}{\thinspace ,}

\newcommand{\vr}{\vec{r}}
\newcommand{\vrp}{\vec{r}\,'}
\newcommand{\erf}{\mathrm{erf}}
\newcommand{\vrho}{\vec{\varrho}}
\newcommand{\vrhop}{\vec{\varrho}\, '}
\newcommand{\sign}{\mathrm{sign}}

\newcommand{\Tr}[1]{\mathrm{Tr}[#1]}
\newcommand{\e}{\varepsilon}
\newcommand{\g}{\gamma}

\newcommand{\half}{\frac{1}{2}}
\newcommand{\vnabla}{\vec{\nabla}}


% Use footnotesize in subscripts
\newcommand{\subsc}[2]{#1_{\mbox{\footnotesize #2}}}




% ------------------- main content ----------------------



% ----------------- title -------------------------

\thispagestyle{empty}

\begin{center}
{\LARGE\bf
\begin{spacing}{1.25}
FFM234, Klassisk fysik och vektorfält - Veckans tal
\end{spacing}
}
\end{center}

% ----------------- author(s) -------------------------

\begin{center}
{\bf \href{{http://fy.chalmers.se/subatom/nt/}}{Christian Forssén}, Institutionen för fysik, Chalmers${}^{}$} \\ [0mm]
\end{center}

\begin{center}
% List of all institutions:
\end{center}
    
% ----------------- end author(s) -------------------------

% --- begin date ---
\begin{center}
Aug 10, 2019
\end{center}
% --- end date ---

\vspace{1cm}


% --- begin exercise ---
\begin{doconceexercise}
\refstepcounter{doconceexercisecounter}

\subsection{Variabelseparation - Elektriskt fält inuti sfär}

% Exempel: PLK Kap. 6.1, Uppg. 10
I sfären med radien $r = a$ finns en 
rymdladdning med tätheten 
\begin{equation}
  \rho\left(r, \theta, \varphi\right) = \rho_0 \frac{r}{a} \sin \theta \cos
\varphi,
\end{equation}
och på sfären gäller att $\Phi\left(a,\theta,\varphi\right) = \Phi_0$.
Bestäm den elektrostatiska potentialen $\Phi$ och det elektriska fältet
$\vec{E}$ inuti sfären.

% --- begin hint in exercise ---

\paragraph{Hint.}
\begin{itemize}
\item Randvillkoret är sfäriskt symmetriskt, medan källan har ett vinkelberoende av formen $\sin \theta \cos \varphi$. Ansätt därför en lösning som består av två delar, vilka var och en har en av dessa egenskaper: $\Phi = f\left(r\right) + g\left(r\right) \sin \theta \cos \varphi$. Notera att randvillkoret ger att $g(a) = 0$.

\item En svårighet är att finna de separerade ekvationerna. Notera att Poissons ekvation skall gälla för alla värden på $r$, $\theta$ och $\varphi$. 

\item Undvik singulära lösningar.

\item Ekvationen för $g$ har både en partikulär- och en homogenlösning.
\end{itemize}

\noindent
% --- end hint in exercise ---


% --- begin answer of exercise ---
\paragraph{Answer.}
\begin{align}
  \vec{E} &= - \nabla \Phi = - \left(\frac{\partial \Phi}{\partial r} 
{\bf \hat r} + \frac{1}{r} \frac{\partial \Phi}{\partial \theta} 
{\bf \hat \theta} + \frac{1}{r \sin \theta} 
\frac{\partial \Phi}{\partial \varphi} {\bf \hat \varphi}\right)
\nonumber \\
&= - \frac{1}{10}\frac{\rho_0}{\epsilon_0} a \left[-3 \frac{r^2}{a^2} \sin \theta
\cos \varphi {\bf \hat r} + \left(\frac{a}{r} - \frac{r^2}{a^2}\right) \left(
\cos \theta \cos \varphi {\bf \hat \theta} - \sin \varphi {\bf \hat \varphi}
\right) \right]
\end{align}

% --- end answer of exercise ---


% --- begin solution of exercise ---
\paragraph{Solution.}
Poissons ekvation för det elektriska fältet $\nabla \vec{E} = \rho / \epsilon_0$ ger oss att
\begin{equation}
  \nabla^2 \Phi = - \frac{\rho_0}{\epsilon_0} \frac{r}{a} \sin \theta \cos
\varphi
\end{equation}
med randvillkoret att $\Phi\left(a,\theta,\varphi\right) = \Phi_0$.
Vi ser här att randvillkoret är sfäriskt symmetriskt, men att 
källan har ett vinkelberoende av formen $\sin \theta \cos \varphi$.  Vi
ansätter därför en lösning som består av två delar, vilka
var och en har en av dessa egenskaper
\begin{equation}
  \Phi = f\left(r\right) + g\left(r\right) \sin \theta \cos \varphi.
\end{equation}
Om vi applicerar Laplace-operatorn på $\Phi$ så får vi
\begin{align}
  \frac{1}{r^2} & \frac{\partial}{\partial r}\left(r^2
\frac{\partial \Phi}{\partial r}\right) + \frac{1}{r^2\sin \theta} 
\frac{\partial}{\partial \theta}\left(\sin \theta 
\frac{\partial \Phi}{\partial \theta}\right) + \frac{1}{r^2 \sin^2 \theta}
\frac{\partial^2 \Phi}{\partial \varphi^2} \nonumber \\
&= \frac{1}{r^2} 
\frac{\partial}{\partial r}\left[r^2\left(f' + g' \sin\theta \cos \varphi\right)
\right] \nonumber \\
& \quad + \frac{1}{r^2 \sin \theta} \frac{\partial}{\partial \theta} \left[
\sin \theta\left( g \cos \theta \cos\varphi\right)\right]
+ \frac{1}{r^2 \sin^2 \theta} \left(-g \sin \theta \cos \varphi\right)
\nonumber \\
&=
\frac{1}{r^2}\left[2r\left(f' + g' \sin \theta \cos \varphi\right) + r^2
\left(f'' + g'' \sin \theta \cos \varphi\right)\right] \nonumber \\
& \quad + \frac{1}{r^2 \sin \theta} \left[g \cos \varphi \left(\cos^2 \theta - \sin^2
\theta\right)\right]- \frac{g}{r^2} \frac{\cos \varphi}{\sin \theta}
\nonumber \\
&= 2 \frac{f'}{r} + f'' + \left(2 \frac{g'}{r} + g''\right) \sin \theta
\cos \varphi + \frac{g}{r^2}\cos \varphi \left( 
\frac{\cos^2 \theta}{\sin \theta} - \sin \theta - \frac{1}{\sin \theta}\right).
\end{align}
Vi beräknar uttrycket i den sista parentesen
\begin{equation}
  \frac{\cos^2 \theta}{\sin \theta} - \sin \theta - \frac{1}{\sin \theta} =
\frac{\cos^2 \theta - \sin^2 \theta - \cos^2 \theta - \sin^2 \theta}
{\sin \theta} = -2 \sin \theta.
\end{equation}
Alltså kan vi skriva Poissons ekvation som
\begin{equation}
  \nabla^2 \Phi = 2 \frac{f'}{r} + f'' + \left(g'' + 2 \frac{g'}{r} - 
2\frac{g}{r^2}\right) \sin \theta \cos \varphi = - \frac{\rho_0}{\epsilon_0}
\frac{r}{a} \sin \theta \cos \varphi.
\end{equation}
Eftersom den här ekvationen skall gälla för alla värden på $r$, $\theta$
och $\varphi$ så ger oss detta ekvationerna
\begin{equation}
  \left\{ \begin{array}{lcl}
f'' + 2\frac{f'}{r} & = & 0\\
g'' + 2 \frac{g'}{r} - 2\frac{g}{r^2} & = & - \frac{\rho_0}{\epsilon_0} 
\frac{r}{a}\\
\end{array}\right.
\end{equation}

Vi börjar med att lösa ekvationen för $f$ genom att ansätta en 
lösning på formen $f =A r^\nu$, men konstanten $A$ kan vi utelämna
för tillfället, så att $f' = \nu r^{\nu-1}$ och
$f'' = \nu (\nu-1) r^{\nu-2}$.  Detta ger oss ekvationen
\begin{equation}
  \nu \left(\nu-1\right) r^{\nu-2}+2\nu r^{\nu-2} = 0,
\end{equation}
som förenklas till
\begin{equation}
  \nu^2 + \nu = 0,
\end{equation}
som har lösningarna $\nu = 0$ och $\nu = -1$.  Därför får vi
\begin{equation}
  f\left(r\right) = A + \frac{B}{r}.
\end{equation}
Här måste vi sätta $B = 0$, för att undvika att potentialen blir
oändlig för $r = 0$, och $A = \Phi_0$ för att vi ska uppfylla 
randvillkoret $\Phi(a,\theta,\varphi) = \Phi_0$.

Vi går nu över till ekvationen för $g$ och börjar med att bestämma
en partikulärlösning på formen $g = Cr^3$.  För denna lösning
har vi $g' = 3Cr^2$ och $g'' = 6Cr$, vilket ger oss
\begin{equation}
  6Cr + 6Cr -2Cr = - \frac{\rho_0}{\epsilon_0} \frac{r}{a},
\end{equation}
som har lösningen
\begin{equation}
  C = - \frac{1}{10} \frac{\rho_0}{\epsilon_0 a},
\end{equation}
och partikulärlösningen är alltså
\begin{equation}
  g\left(r\right) = - \frac{1}{10} \frac{\rho_0}{\epsilon_0} \frac{r^3}{a}.
\end{equation}
Vi behöver också bestämma lösningen till den homogena ekvationen,
och därför ansätter vi $g(r) = Dr^\nu$, men för ögonblicket 
utelämnar vi konstanten $D$.  Eftersom $g' = \nu r^{\nu-1}$ och $g'' = \nu
(\nu-1) r^{\nu-2}$, så får vi
\begin{equation}
  \nu \left(\nu-1\right) r^{\nu-2} + 2\nu r^{\nu-2} -2r^{\nu-2} = 0,
\end{equation}
vilket ger oss andragradsekvationen
\begin{equation}
  \nu^2 + \nu -2 = 0.
\end{equation}
Denna har lösningarna
\begin{equation}
  \nu = - \frac{1}{2} \pm \sqrt{\left(\frac{1}{2}\right)^2+2} = -\frac{1}{2}
\pm \frac{3}{2}.
\end{equation}
Alltså har vi $\nu = -2$ och $\nu = 1$, så den allmänna lösningen 
för $g$ blir
\begin{equation}
  g\left(r\right) = \frac{D}{r^2} + Er - \frac{1}{10} \frac{\rho_0}{\epsilon_0}
\frac{r^3}{a}.
\end{equation}
Vi sätter $D = 0$ eftersom potentialen inte kan bli singulär i $r = 0$.
Vårt randvillkor vid $r = a$ säger oss å andra sidan att $g(a) = 0$,
vilket ger oss
\begin{equation}
  0 = Ea - \frac{1}{10} \frac{\rho_0}{\epsilon_0}a^2,
\end{equation}
så att
\begin{equation}
  E = \frac{1}{10} \frac{\rho_0}{\epsilon_0} a
\end{equation}
Alltså är potentialen
\begin{equation}
  \Phi\left(r,\theta,\varphi\right) = \Phi_0 + \frac{1}{10} 
\frac{\rho_0}{\epsilon_0} a \left(a - \frac{r^3}{a^2}\right) \sin \theta
\cos \varphi.
\end{equation}
Vi kan nu beräkna det elektriska fältet ur
\begin{align}
  \vec{E} &= - \nabla \Phi = - \left(\frac{\partial \Phi}{\partial r} 
{\bf \hat r} + \frac{1}{r} \frac{\partial \Phi}{\partial \theta} 
{\bf \hat \theta} + \frac{1}{r \sin \theta} 
\frac{\partial \Phi}{\partial \varphi} {\bf \hat \varphi}\right)
\nonumber \\
&= - \frac{1}{10}\frac{\rho_0}{\epsilon_0} a \left[-3 \frac{r^2}{a^2} \sin \theta
\cos \varphi {\bf \hat r} + \left(\frac{a}{r} - \frac{r^2}{a^2}\right) \left(
\cos \theta \cos \varphi {\bf \hat \theta} - \sin \varphi {\bf \hat \varphi}
\right) \right]
\end{align}

% --- end solution of exercise ---

% Closing remarks for this Exercise

\paragraph{Remarks.}
Uppgiften illustrerar variabelseparation i tre dimensioner och styrkan med en bra lösningsansats. Lösningen innehåller ganska många steg och uppgiften kan därför klassificeras som svår. Ta gärna en titt på lösningen om du kör fast.


\end{doconceexercise}
% --- end exercise ---


% ------------------- end of main content ---------------

% #ifdef PREAMBLE
\end{document}
% #endif

