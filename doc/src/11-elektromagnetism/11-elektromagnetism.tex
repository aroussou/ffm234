%%
%% Automatically generated file from DocOnce source
%% (https://github.com/hplgit/doconce/)
%%

% #define PREAMBLE

% #ifdef PREAMBLE
%-------------------- begin preamble ----------------------

\documentclass[%
oneside,                 % oneside: electronic viewing, twoside: printing
final,                   % draft: marks overfull hboxes, figures with paths
10pt]{article}

\listfiles               %  print all files needed to compile this document

\usepackage{relsize,makeidx,color,setspace,amsmath,amsfonts,amssymb}
\usepackage[table]{xcolor}
\usepackage{bm,ltablex,microtype}

\usepackage[pdftex]{graphicx}

\usepackage[T1]{fontenc}
%\usepackage[latin1]{inputenc}
\usepackage{ucs}
\usepackage[utf8x]{inputenc}

\usepackage{lmodern}         % Latin Modern fonts derived from Computer Modern

% Hyperlinks in PDF:
\definecolor{linkcolor}{rgb}{0,0,0.4}
\usepackage{hyperref}
\hypersetup{
    breaklinks=true,
    colorlinks=true,
    linkcolor=linkcolor,
    urlcolor=linkcolor,
    citecolor=black,
    filecolor=black,
    %filecolor=blue,
    pdfmenubar=true,
    pdftoolbar=true,
    bookmarksdepth=3   % Uncomment (and tweak) for PDF bookmarks with more levels than the TOC
    }
%\hyperbaseurl{}   % hyperlinks are relative to this root

\setcounter{tocdepth}{2}  % levels in table of contents

% newcommands for typesetting inline (doconce) comments
\newcommand{\shortinlinecomment}[3]{{\color{red}{\bf #1}: #2}}
\newcommand{\longinlinecomment}[3]{{\color{red}{\bf #1}: #2}}

\usepackage[framemethod=TikZ]{mdframed}

% --- begin definitions of admonition environments ---

% Admonition style "mdfbox" is an oval colored box based on mdframed
% "notice" admon
\definecolor{mdfbox_notice_background}{rgb}{1,1,1}
\newmdenv[
  skipabove=15pt,
  skipbelow=15pt,
  outerlinewidth=0,
  backgroundcolor=mdfbox_notice_background,
  linecolor=black,
  linewidth=2pt,       % frame thickness
  frametitlebackgroundcolor=mdfbox_notice_background,
  frametitlerule=true,
  frametitlefont=\normalfont\bfseries,
  shadow=false,        % frame shadow?
  shadowsize=11pt,
  leftmargin=0,
  rightmargin=0,
  roundcorner=5,
  needspace=0pt,
]{notice_mdfboxmdframed}

\newenvironment{notice_mdfboxadmon}[1][]{
\begin{notice_mdfboxmdframed}[frametitle=#1]
}
{
\end{notice_mdfboxmdframed}
}

% Admonition style "mdfbox" is an oval colored box based on mdframed
% "summary" admon
\definecolor{mdfbox_summary_background}{rgb}{1,1,1}
\newmdenv[
  skipabove=15pt,
  skipbelow=15pt,
  outerlinewidth=0,
  backgroundcolor=mdfbox_summary_background,
  linecolor=black,
  linewidth=2pt,       % frame thickness
  frametitlebackgroundcolor=mdfbox_summary_background,
  frametitlerule=true,
  frametitlefont=\normalfont\bfseries,
  shadow=false,        % frame shadow?
  shadowsize=11pt,
  leftmargin=0,
  rightmargin=0,
  roundcorner=5,
  needspace=0pt,
]{summary_mdfboxmdframed}

\newenvironment{summary_mdfboxadmon}[1][]{
\begin{summary_mdfboxmdframed}[frametitle=#1]
}
{
\end{summary_mdfboxmdframed}
}

% Admonition style "mdfbox" is an oval colored box based on mdframed
% "warning" admon
\definecolor{mdfbox_warning_background}{rgb}{1,1,1}
\newmdenv[
  skipabove=15pt,
  skipbelow=15pt,
  outerlinewidth=0,
  backgroundcolor=mdfbox_warning_background,
  linecolor=black,
  linewidth=2pt,       % frame thickness
  frametitlebackgroundcolor=mdfbox_warning_background,
  frametitlerule=true,
  frametitlefont=\normalfont\bfseries,
  shadow=false,        % frame shadow?
  shadowsize=11pt,
  leftmargin=0,
  rightmargin=0,
  roundcorner=5,
  needspace=0pt,
]{warning_mdfboxmdframed}

\newenvironment{warning_mdfboxadmon}[1][]{
\begin{warning_mdfboxmdframed}[frametitle=#1]
}
{
\end{warning_mdfboxmdframed}
}

% Admonition style "mdfbox" is an oval colored box based on mdframed
% "question" admon
\definecolor{mdfbox_question_background}{rgb}{1,1,1}
\newmdenv[
  skipabove=15pt,
  skipbelow=15pt,
  outerlinewidth=0,
  backgroundcolor=mdfbox_question_background,
  linecolor=black,
  linewidth=2pt,       % frame thickness
  frametitlebackgroundcolor=mdfbox_question_background,
  frametitlerule=true,
  frametitlefont=\normalfont\bfseries,
  shadow=false,        % frame shadow?
  shadowsize=11pt,
  leftmargin=0,
  rightmargin=0,
  roundcorner=5,
  needspace=0pt,
]{question_mdfboxmdframed}

\newenvironment{question_mdfboxadmon}[1][]{
\begin{question_mdfboxmdframed}[frametitle=#1]
}
{
\end{question_mdfboxmdframed}
}

% Admonition style "mdfbox" is an oval colored box based on mdframed
% "block" admon
\definecolor{mdfbox_block_background}{rgb}{1,1,1}
\newmdenv[
  skipabove=15pt,
  skipbelow=15pt,
  outerlinewidth=0,
  backgroundcolor=mdfbox_block_background,
  linecolor=black,
  linewidth=2pt,       % frame thickness
  frametitlebackgroundcolor=mdfbox_block_background,
  frametitlerule=true,
  frametitlefont=\normalfont\bfseries,
  shadow=false,        % frame shadow?
  shadowsize=11pt,
  leftmargin=0,
  rightmargin=0,
  roundcorner=5,
  needspace=0pt,
]{block_mdfboxmdframed}

\newenvironment{block_mdfboxadmon}[1][]{
\begin{block_mdfboxmdframed}[frametitle=#1]
}
{
\end{block_mdfboxmdframed}
}

% --- end of definitions of admonition environments ---

% prevent orhpans and widows
\clubpenalty = 10000
\widowpenalty = 10000

% --- end of standard preamble for documents ---


\usepackage[swedish]{babel}

\raggedbottom
\makeindex
\usepackage[totoc]{idxlayout}   % for index in the toc
\usepackage[nottoc]{tocbibind}  % for references/bibliography in the toc

%-------------------- end preamble ----------------------

\begin{document}

% matching end for #ifdef PREAMBLE
% #endif

\newcommand{\exercisesection}[1]{\subsection*{#1}}

% This file is to be run by preprocess to produce newcommands.tex
% to be included in .tex files.
% There are format-specific tests here for the newcommands (i.e.,
% different definitions of the commands depending on latex or mathjax).

% Newcommands for LaTeX math.
\newcommand{\tp}{\thinspace .}
\renewcommand{\Re}{\bbbr}
\newcommand{\Oof}[1]{\mathcal{O}(#1)}
\newcommand{\Prob}[1]{\hbox{P}(#1)}
\newcommand{\Var}[1]{\hbox{Var}(#1)}
\newcommand{\Cov}[2]{\hbox{Cov}(#1,#2)}
\newcommand{\StDev}[1]{\hbox{StDev}(#1)}

\newcommand{\punkt}{\thinspace .}
\newcommand{\komma}{\thinspace ,}

\newcommand{\vr}{\vec{r}}
\newcommand{\vrp}{\vec{r}\,'}
\newcommand{\erf}{\mathrm{erf}}
\newcommand{\vrho}{\vec{\varrho}}
\newcommand{\vrhop}{\vec{\varrho}\, '}
\newcommand{\sign}{\mathrm{sign}}

\newcommand{\Tr}[1]{\mathrm{Tr}[#1]}
\newcommand{\e}{\varepsilon}
\newcommand{\g}{\gamma}

\newcommand{\half}{\frac{1}{2}}
\newcommand{\vnabla}{\vec{\nabla}}


% Use footnotesize in subscripts
\newcommand{\subsc}[2]{#1_{\mbox{\footnotesize #2}}}




% ------------------- main content ----------------------



% ----------------- title -------------------------

\thispagestyle{empty}

\begin{center}
{\LARGE\bf
\begin{spacing}{1.25}
FFM234, Klassisk fysik och vektorfält - Föreläsningsanteckningar
\end{spacing}
}
\end{center}

% ----------------- author(s) -------------------------

\begin{center}
{\bf \href{{http://fy.chalmers.se/subatom/tsp/}}{Christian Forssén}, Institutionen för fysik, Chalmers, Göteborg, Sverige${}^{}$} \\ [0mm]
\end{center}

\begin{center}
% List of all institutions:
\end{center}
    
% ----------------- end author(s) -------------------------

% --- begin date ---
\begin{center}
Oct 15, 2019
\end{center}
% --- end date ---

\vspace{1cm}


\section*{11. Elektromagnetiska fält och Maxwells ekvationer}

Vi startar med att betrakta \emph{statiska} elektriska och magnetiska (elektrostatik och magnetostatik) för att sedan ta med \emph{tidsberoendet} och se hur det innebär en koppling mellan de två fälten.

\subsection*{Maxwells ekvationer (tidsoberoende fält)}


\paragraph{Elektrostatik.}


\vspace{3mm}


Statiska elektriska fält $\vec{E}(\vec{r})$ uppfyller
\begin{align}
  \vnabla \cdot \vec{E} &= \frac{\rho}{\epsilon_0} \\
  \vnabla \times \vec{E} &= 0
\end{align}
\begin{itemize}
\item $\rho(\vec{r})$ = elektrisk laddningstäthet.

\item $\epsilon_0$ = dielektricitetskonstant i vakuum
\end{itemize}

\noindent
Den andra ekvationen säger att elektrostatiska fält är rotationsfria ($\vnabla \times \vec{E} = 0$) och därmed konservativa. 
$$
\vec{E} = -\vnabla \phi,
$$
där $\phi$ är den elektrostatiska potentialen. 

Den första är Gauss lag och säger att elektriska fält kan ha elektriska
laddningar som källor. Den elektrostatiska potentialen uppfyller därmed Poissons ekvation
$$
\Delta \phi = - \frac{\rho}{\epsilon_0}
$$

\paragraph{Magnetostatik.}


\vspace{3mm}


Statiska magnetiska fält $\vec{B}(\vec{r})$ uppfyller
\begin{align}
  \vnabla \cdot \vec{B} &= 0 \\
  \vnabla \times \vec{B} &= \mu_0 \vec{\jmath}
\end{align}
där den första säger att det inte finns några magnetiska laddningar och den andra är Amperes lag.
\begin{itemize}
\item $\vec{\jmath}(\vec{r})$ = elektrisk strömtäthet.

\item $\mu_0$ = magnetisk permeabilitet i vakuum
\end{itemize}

\noindent
Den första ekvationen säger att magnetostatiska fält är divergensfria ($\vnabla \cdot \vec{B} = 0$) (eller källfritt) och kan uttryckas med en vektorpotential
$$
\vec{B} = \vnabla \times \vec{A},
$$
där Gaugeinvarians innebär att vektorpotentialen inte är fullständigt
bestämd
$$
\vec{A}(\vec{r}) \mapsto \vec{A}(\vec{r}) + \vnabla \Lambda(\vec{r}).
$$
Detta gör det möjligt att välja Gaugeparameter så att $\vnabla \cdot \vec{A} = 0$ och vektorpotentialen uppfyller Poissons ekvation
$$
\Delta \vec{A} = -\mu_0 \vec{\jmath}.
$$


\begin{warning_mdfboxadmon}[SI enheter]
\begin{align}
\mu_0 &= 4 \pi \cdot 10^{-7} \; \mathrm{T}\mathrm{A}^{-1}\mathrm{m}^{-1} \nonumber \\
\mu_0 \epsilon_0 &= \frac{1}{c^2} \nonumber \\
c &= 299\,792\,458 \; \mathrm{m}\mathrm{s}^{-1} \nonumber
\end{align}
\end{warning_mdfboxadmon} % title: SI enheter




\begin{notice_mdfboxadmon}[Exempel: Bestämning av elektriskt fält]

En elektrisk laddning $Q$ är jämnt fördelad i en sfär med radien $a$. Den omges av ett tunt sfäriskt skal med radien 2a och laddningen $-Q$. Bestäm det elektriska fältet $\vec{E}(\vec{r})$ och potentialen $\phi(\vec{r})$ överallt.

\paragraph{Lösning:}
På grund av att laddningsfördelningen har sfärisk
symmetri, så blir $E_\varphi = E_\theta = 0$, och $E_r$ beror inte på
$\theta$ och $\varphi$.  Vi kan då beräkna $E_r$ med hjälp av Gauss
lag genom att införa en sfärisk volym med radie $r$ och begränsningsyta $S_r$. Ytelementet blir $d\vec{S} = \hat{r} r^2 \sin\theta d\theta d\varphi$. Gauss lag för det elektriska fältet blir
\begin{equation}
  \oint_{S_r} \vec{E} \cdot \mbox{d}\vec{S} = \iint E_r r^2 \sin\theta d\theta d\varphi 
  = \frac{1}{\epsilon_0} \int_V \rho(r) dV = \frac{Q_r}{\epsilon_0},
\end{equation}
där $\rho(r) = \epsilon_0 \vec\nabla\cdot\vec E$ är laddningstätheten och $Q_r$ alltså är den inneslutna laddningen.

Om vi börjar med fallet att sfären har en radie $r > 2a$, så ser vi 
att den totala inneslutna laddningen är $Q - Q = 0$.  Alltså har
vi att 
\begin{equation}
  \oint_{S_r} \vec{E} \cdot \mbox{d}\vec{S} = \iint E_r r^2 \sin\theta d\theta d\varphi 
  = 0,
\end{equation}
och av detta följer att att $E_r = 0$.

Om sfären har en radie $r$ så att $a < r < 2a$, så är den laddningen,
som $S$ innesluter, $Q$.  Då ger oss Gauss lag att
\begin{equation}
  \oint_{S_r} \vec{E} \cdot \mbox{d}\vec{S} = \frac{Q}{\epsilon_0}.
\end{equation}
Integralen i vänsterledet har värdet
\begin{equation}
  \oint_{S_r} \vec{E} \cdot \mbox{d}\vec{S} = \iint E_r r^2 \sin\theta d\theta d\varphi = 4\pi r^2 E_r,
\end{equation}
och vi kan lösa ut
\begin{equation}
  E_r = \frac{1}{4\pi \epsilon_0} \frac{Q}{r^2}.
\end{equation}

Slutligen har vi fallet att sfärens radie $r < a$.  Eftersom laddningen $Q$
är jämt fördelad över volymen, så innebär det att sfären innesluter
laddningen
\begin{equation}
  Q \left(\frac{r}{a}\right)^3.
\end{equation}
Gauss lag ger oss allts\aa
\begin{equation}
  \oint_{S_r} \vec{E} \cdot \mbox{d}\vec{S} = \frac{Q}{\epsilon_0}
\left(\frac{r}{a}\right)^3,
\end{equation}
vilket blir
\begin{equation}
  4\pi r^2 E_r = \frac{Q}{\epsilon_0}\left(\frac{r}{a}\right)^3.
\end{equation}
Vi kan nu lösa ut 
\begin{equation}
  E_r = \frac{1}{4\pi \epsilon_0} \frac{Qr}{a^3}.
\end{equation}

Om vi nu sammanfattar våra resultat, så har vi för det elektriska
fältet
\begin{equation}
  E_r = \left\{\begin{array}{ll}
\frac{1}{4\pi \epsilon_0} \frac{Qr}{a^3} \quad & r < a\\
\frac{1}{4\pi \epsilon_0} \frac{Q}{r^2} & a < r < 2a\\
0 & r > 2a\\
\end{array}\right.
\end{equation}

Vi skall nu utnyttja detta för att bestämma potentialen, för vilken det
gäller att $\vec{E} = - \vnabla \phi$.  Eftersom enbart den radiella komponenten av E-fältet är nollskild har vi att $E_r = -\partial_r \phi$. I vårt fall kan vi uttrycka 
potentialen som integralen
\begin{equation}
  \int_r^\infty E_r \mbox{d}r = -\int_r^\infty \partial_r \phi \mbox{d}r = \phi\left(r\right) - \phi(\infty).
\end{equation}
Vi sätter potentialen till 0 i oändligheten och börjar med att bestämma potentialen för intervallet
$r > 2a$.  Eftersom $E_r = 0$ här så följer det att också potentialen
blir 0.  

Potentialen skall överallt vara kontinuerlig, vilket inte behöver vara
sant för $E$-fältet. När vi går till
intervallet $a \le r \le 2a$ så kan den övre integrationsgränsen
sättas till $2a$, eftersom potentialen är 0 utanför.
\begin{equation}
  \phi\left(r\right) = \int_r^{2a} \frac{1}{4\pi \epsilon_0} \frac{Q}{r^2}
\mbox{d}r = \frac{Q}{4\pi\epsilon_0}\left[- \frac{1}{r}\right]_r^{2a} =
\frac{Q}{4\pi\epsilon_0} \left(\frac{1}{r}-\frac{1}{2a}\right).
\end{equation}
Speciellt så lägger vi märke till att potentialen i punkten $r = a$
blir
\begin{equation}
  \phi \left(a\right) = \frac{Q}{8\pi \epsilon_0 a}.
\end{equation}
Det enklaste sättet att garantera att $\phi$ blir kontinuerlig vid $r = a$
är nu att skriva potentialen för $r\le a$ som
\begin{align}
  \phi \left(r\right) = \int_r^\infty E_r \mbox{d}r = 
\frac{Q}{8\pi \epsilon_0 a} + \int_r^a \frac{Qr}{4\pi \epsilon_0 a^3} 
\mbox{d}r = \frac{Q}{8\pi \epsilon_0 a} + 
\left[\frac{Qr^2}{8\pi \epsilon_0 a^3}\right]_r^a = \nonumber \\
\frac{Q}{8\pi \epsilon_0 a} 
+  \frac{Q}{8\pi \epsilon_0 a} - \frac{Qr^2}{8\pi \epsilon_0 a^3} = 
\frac{Q}{4\pi \epsilon_0 a} - \frac{Qr^2}{8\pi \epsilon_0 a^3} =
\frac{Q}{8\pi \epsilon_0} \frac{2a^2 -r^2}{a^3}.
\end{align}
Vi kan till slut sammanfatta potentialen som
\begin{equation}
  \phi\left(r\right) = \left\{\begin{array}{lr}
\frac{Q}{8\pi \epsilon_0} \frac{2a^2 -r^2}{a^3} & r\le a\\
\frac{Q}{4\pi \epsilon_0} \left(\frac{1}{r} - \frac{1}{2a}\right) & a\le r \le 
2a\\
0 & r \ge 2a\\
\end{array}
\right.
\end{equation}
\end{notice_mdfboxadmon} % title: Exempel: Bestämning av elektriskt fält





\begin{notice_mdfboxadmon}[Exempel: Bestämning av ett magnetfält]

En oändligt lång rak ledare har ett cirkulärt tvärsnitt med radien
$a$ och leder en likström med strömstyrkan $I$.  Använd Amperes lag 
för att härleda magnetfältet i och kring ledaren om materialet i den
antas homogent och isotropt.


\vspace{3mm}


\emph{Lösning:}  Det elektriska fältet är stationärt i detta fall och vi kan då använda Amperes lag utan någon förskjutningsström
\begin{equation}
  \int_{S_\rho} \vnabla \times \vec{B} \cdot \mbox{d}\vec{S} = \mu_0 \int_{S_\rho} \vec{\jmath} \cdot \mbox{d}\vec{S} = \mu_0 I_\rho,
\end{equation}
där $I_\rho$ är strömmen som passerar genom en tvärsnittsyta $S_\rho$ som är en cirkelskiva med $z$-axeln som centrum och radien $\rho$. Använder vi Stokes sats får vi
\begin{equation}
  \oint_{\partial S_\rho} \vec{B} \cdot \mbox{d}\vec{r} = \mu_0 I_\rho,
\end{equation}
där vi låter $\partial S_\rho$ vara randen till $S_\rho$, dvs en cirkel som genomlöps moturs. Först 
tittar vi på fallet att cirkelns radie $\rho > a$.  Strömmen 
är då $I$, och Amperes lag ger oss
\begin{equation}
  \oint_{\partial S_\rho} \vec{B} \cdot \mbox{d}\vec{r} = \mu_0 I.
\end{equation}
Vi vet att om den elektriska ledaren sammanfaller med  $z$-axeln så är magnetfältet riktat i $\varphi$-riktningen.
Integralen blir då
\begin{equation}
  \oint_{\partial S_\rho} \vec{B} \cdot \mbox{d}\vec{r} = 2\pi \rho B_\varphi.
\end{equation}
Vi kan nu lösa ut
\begin{equation}
  B_\varphi = \frac{\mu_0 I}{2\pi \rho}.
\end{equation}

I fallet att cirkelns radie $\rho < a$ så antar vi att strömmen är 
jämnt fördelad i tråden, vilket ger oss att den inneslutna strömmen
blir
\begin{equation}
  I_\rho = I \left(\frac{\rho}{a}\right)^2.
\end{equation}
Integralen blir nu
\begin{equation}
  \oint_{\partial S_\rho} \vec{B} \cdot \mbox{d}\vec{r} = 2\pi \rho B_\varphi = \mu_0 I 
\left(\frac{\rho}{a}\right)^2.
\end{equation}
Vi kan då lösa 
\begin{equation}
  B_\varphi = \frac{\mu_0 I}{2\pi} \frac{\rho}{a^2}.
\end{equation}
Sammanfattningsvis har vi alltså att $\vec{B} = B_\varphi \hat{e}_\varphi$ med
\begin{equation}
  B_\varphi = \left\{\begin{array}{lr}
\frac{\mu_0 I}{2\pi \rho} & \rho > a\\
\frac{\mu_0 I\rho}{2\pi a^2} & \rho \le a\\
\end{array}\right.
\end{equation}
\end{notice_mdfboxadmon} % title: Exempel: Bestämning av ett magnetfält





\begin{notice_mdfboxadmon}[Exempel: Bestämning av vektorpotentialen]

Betrakta en elektrisk ledare parallell med $z$-axeln.  Genom ledaren flyter en ström $I$.  Då omges ledaren av ett magnetfält
$$
\vec{B} = \frac{\mu_0}{2\pi}\frac{I}{\rho}{\hat \varphi}.
$$
Bestäm vektorpotentialen $\vec{A}$ och finn den differentialekvation som beskriver detta fält.

\shortinlinecomment{Kommentar 1}{ Notera att magnetfältet uppvisar en singularitet. Känns den igen? Det är fältet från en virveltråd. }{ Notera att magnetfältet uppvisar }

Notera att
$$
\vec{B} = \vnabla \times \vec{A} = \frac{1}{\rho}
\begin{vmatrix}
\hat{\rho} & \rho \hat{\varphi} & \hat{z} \\
\partial_\rho & \partial_\varphi & \partial_z \\
A_\rho & {\rho A_\varphi} & A_z
\end{vmatrix} 
$$
Vi kan därför bestämma vektorpotentialen ur ekvationerna
\begin{equation}
  \frac{1}{\rho}\frac{\partial A_z}{\partial \varphi} - \frac{\partial A_\varphi}
{\partial z} = 0
\label{rho}
\end{equation}
\begin{equation}
  \frac{\partial A_\rho}{\partial z} - \frac{\partial A_z}{\partial \rho} =
\frac{\mu_0}{2\pi}\frac{I}{\rho}
\label{phi}
\end{equation}
och
\begin{equation}
  \frac{1}{\rho}\left[\frac{\partial}{\partial \rho} \left(\rho A_\varphi\right)
- \frac{\partial A_\rho}{\partial \varphi}\right] = 0.
\label{z}
\end{equation}
Vi skall finna en vektorpotential så att dessa ekvationer uppfylles. Vi provar med $A_\rho = A_\varphi=0$ och $A_z \neq 0$. Denna ansatz ger
\begin{equation}
  \vec{A} = - \frac{\mu_0}{2\pi} I \log\frac{\rho}{\rho_0} { \hat z},
\end{equation}
där $\rho_0$ är en godtycklig konstant.

Låt oss nu betrakta Amperes lag i det tidsoberoende fallet
\begin{equation}
  \mu_0 \vec{\jmath} =  \vnabla \times \vec{B}.
\end{equation}
Om vi nu ersätter $\vec{B}$ med $\vnabla \times \vec{A}$ så har vi
\begin{equation}
  \vnabla \times \vnabla \times \vec{A} = \mu_0\vec{\jmath}.
\end{equation}
För vänsterledet har vi räknereglen
\begin{equation}
  \vnabla \times \vnabla \times \vec{A} =
\vnabla\left(\vnabla\cdot\vec{A}\right) -\vnabla^2\vec{A}
\end{equation}
Den frihet, gauge, som vi har i att bestämma $\vec{A}$ gör det alltid möjligt att garantera att $\vnabla \cdot \vec{A} = 0$, så vi kan reducera ekvationen till
\begin{equation}
  \vnabla^2 \vec{A} = -\mu_0 \vec{\jmath},
\end{equation}
och vi har på så sätt kommit fram till en Poisson-ekvation för vektorpotentialen. Notera att gauge-valet $\vnabla \cdot \vec{A} = 0$ faktiskt är uppfyllt för den vektorpotential som vi konstruerade ovan.
\end{notice_mdfboxadmon} % title: Exempel: Bestämning av vektorpotentialen



\subsection*{Maxwells ekvationer}

\longinlinecomment{Kommentar 2}{ Maxwell satte 1864 upp fyra stycken ekvationer som gav en fullständig beskrivning av ett elektromagnetiskt fält.  Dock, som vi skall se, inskränkte sig hans eget bidrag till en term i en av ekvationerna. }{ Maxwell satte 1864 upp }

För \emph{tidsberoende} fält finns det en koppling mellan elektriska och magnetiska fält. 

\paragraph{EM koppling 1: Kontinuitetsekvationen (konservering av elektrisk laddning).}
Låt oss betrakta sambandet mellan elektrisk strömtäthet och (rotationen av) ett magnetfält. Detta samband har konsekvenser för kontinuitetsekvationen för elektrisk laddning
$$
  \frac{\partial \rho}{\partial t} = - \vnabla \cdot \vec{\jmath}.
$$
från Amperes lag får vi nämligen att HL i kontinuitetsekvationen blir
$$
  \vnabla \cdot \vec{\jmath} = \frac{1}{\mu_0} \vnabla \cdot \left(\vnabla \times
\vec{B}\right) = 0,
$$
enligt räknereglerna för vektoroperatorerna. Detta skulle betyda att
$$
  \frac{\partial \rho}{\partial t} = 0,
$$
vilket är orimligt! 
För det betyder att det inte går att flytta en elektrisk laddning.  

Maxwells lösning var att lägga till en term (förskjutningsströmmen)
\begin{equation}
  \vnabla \times \vec{B} = \mu_0 \vec{\jmath} + \mu_0 \epsilon_0 \frac{\partial \vec{E}}{\partial t}.
\end{equation}

Notera att den extra termen betyder att kontinuitetsekvationen uppfylls eftersom
\[
-\vnabla \cdot \vec{\jmath} = -\frac{1}{\mu_0} \vnabla \cdot \left(\vnabla \times
\vec{B}\right) + \epsilon_0 \frac{\partial (\vnabla \cdot \vec{E})}{\partial t}
= \frac{\partial \rho}{\partial t},
\]
där vi har utnyttjat Gauss lag.

\paragraph{EM koppling 2: Induktion (Faradays lag).}
Vi har sedan tidigare funnit att elektriska laddningar kan skapa elektriska fält  och att elektriska strömmar kan skapa magnetfält.

Vi vet dock att elektriska fält också kan skapas genom induktion.  En förändring av det magnetiska flödet, $\Phi$, genom en elektrisk ledare inducerar en spänning, $U$, i ledaren då det magnetiska flödet förändras.

\begin{equation}
  U = - \frac{\mbox{d}\Phi}{\mbox{d}t},
  \label{eq:faraday}
\end{equation}
där $\Phi$ är ett magnetiskt flöde genom ytan $S$
\begin{equation}
  \Phi = \int_S \vec{B} \cdot \mbox{d} \vec{S}.
\end{equation}
och $U$ är den inducerade spänningen längs randen $\partial S$
\begin{equation}
  U = \oint_{\partial S} \vec{E} \cdot \mbox{d}\vec{r}.
  \label{eq:indU}
\end{equation}
Vi sätter nu Ekv (\ref{eq:indU}) och (\ref{eq:faraday}) lika med varandra och får Faradays lag på integralform
$$
  \oint_{\partial S} \vec{E} \cdot \mbox{d}\vec{r} = -
\int_S \frac{\partial \vec{B}}{\partial t} \cdot \mbox{d}\vec{S}.
$$

Använder vi Stokes sats på VL får vi Faradays lag på differentialform (notera att ytan $S$ är helt godtycklig)
\begin{equation}
  \vnabla \times \vec{E} = - \frac{\partial \vec{B}}{\partial t}.
\end{equation}

\shortinlinecomment{Kommentar 3}{ Det elektriska fältet är inte längre konservativt vilket ju modifierar en av våra ekvationer. }{ Det elektriska fältet är }

Vi får nu alltså att Maxwells ekvationer blir


\begin{summary_mdfboxadmon}[Maxwells ekvationer]
\begin{align}
  \vnabla \cdot \vec{E} &= \frac{\rho}{\epsilon_0}, \\
  \vnabla \times \vec{E} &= - \frac{\partial B}{\partial t}, \\
  \vnabla \cdot \vec{B} &= 0, \\
  \vnabla \times \vec{B} &= \mu_0 \vec{\jmath} + \mu_0 \epsilon_0 
\frac{\partial E}{\partial t}.
\end{align}
\end{summary_mdfboxadmon} % title: Maxwells ekvationer



\begin{itemize}
\item Ett allmänt fält $\vec{E}$ kan vi alltid dela upp i en del som är virvelfri, och en del som är källfri. 

\item För ett elektrisk fält gäller alltså att den virvelfria delen kan skrivas som $-\vnabla \phi$.  Enligt induktionsekvationen är
\end{itemize}

\noindent
$$
\vnabla \times \vec{E} = - \frac{\partial \vec{B}}{\partial t} = -
\frac{\partial (\vnabla \times \vec{A})}{\partial t} = - \vnabla \times 
\frac{\partial \vec{A}}{\partial t}.
$$
Alltså kan vi skriva den källfria delen av $\vec{E}$ som
$-\partial \vec{A} / \partial t$, så att vi totalt har
\begin{equation}
  \vec{E} = - \vnabla \phi - \frac{\partial \vec{A}}{\partial t}.
\end{equation}


\subsection*{Maxwells ekvationer i vakuum och den elektromagnetiska vågekvationen}


\begin{warning_mdfboxadmon}[OBS!]
Resten av dessa anteckningar behandlar kap 11.3-11.7 som ej ingår i kursen. Anteckningarna är endast med för bättre koppling till motsvarande material i andra kurser.
\end{warning_mdfboxadmon} % title: OBS!



\begin{itemize}
\item En mycket viktig konsekvens av Maxwells ekvationer är att det existerar våglösningar.
\end{itemize}

\noindent
I vakuum ($\rho=0$ och $\vec{\jmath} = 0$) blir Maxwells ekvationer
\begin{equation}
  \vnabla \cdot \vec{E} = 0
\end{equation}
\begin{equation}
  \vnabla \times \vec{E} = - \frac{\partial \vec{B}}{\partial t}
\label{induction}
\end{equation}
\begin{equation}
  \vnabla \cdot \vec{B} = 0
\end{equation}
\begin{equation}
  \vnabla \times \vec{B} = \epsilon \mu_0 \frac{\partial \vec{E}}{\partial t}
\label{amp_max}
\end{equation}
Vi kan nu till exempel beräkna rotationen av induktionsekvationen Ekv. (\ref{induction})
\begin{equation}
  \vnabla \times \vnabla \times \vec{E} = \vnabla \left( \vnabla \cdot \vec{E}
\right) - \vnabla^2 \vec{E} = - \vnabla \times 
\frac{\partial \vec{B}}{\partial t} =
- \frac{\partial (\vnabla \times \vec{B})}{\partial t}.
\end{equation}
Vi kan nu utnyttja att $\vnabla \cdot \vec{E} =0$ och Ekv. (\ref{amp_max})
\begin{equation}
  - \vnabla^2 \vec{E} = - \frac{\partial}{\partial t} \epsilon_0 \mu_0
\frac{\partial \vec{E}}{\partial t} = 
- \epsilon_0 \mu_0 \frac{\partial^2 \vec{E}}{\partial t^2}.
\end{equation}
Eftersom $\epsilon_0 \mu_0 = 1/c^2$ kan vi skriva detta
\begin{equation}
  \left( \vnabla^2 - \frac{1}{c^2} \frac{\partial^2}{\partial t^2} \right) \vec{E} = 0,
\label{vaag}
\end{equation}
vilket är en \emph{vågekvation}. 


\begin{notice_mdfboxadmon}[Exempel: Vågekvationen i $D=1$]

Betrakta fältet $\vec{E} = E(x,t)$ i en dimension och motsvarande vågekvation
\begin{equation}
  \left( \frac{\partial^2}{\partial x^2} - \frac{1}{c^2} \frac{\partial^2}{\partial t^2} \right) E(x,t) = 0.
\end{equation}
Ekvationen har då lösningar på formerna $E_+ = E_0 \sin(k x - \omega t)$ och $E_- = E_0 \sin(k x + \omega t)$, vilka beskriver vågor och motiverar varför Ekv. (\ref{vaag}) kallas för vågekvationen.

Med denna ansatz ($E_+$) ger vågekvationen
$$
-k^2 + \frac{\omega^2}{c^2} = 0 \quad \Rightarrow \quad \omega = c |k|,
$$
vilket kallas för en dispersionsrelation.

\shortinlinecomment{Rita 4}{ funktionen $E(x,t)$ med en $x$-axel och en $t$-axel. Illustrera våglängd och periodtid. }{ funktionen $E(x,t)$ med en }

\begin{itemize}
\item Vid given tid $t$: samma fas då $x \mapsto x + \frac{2 \pi}{k}$. 

\item Dvs \emph{våglängden} $\lambda$ relaterar till \emph{vågtalet} $k$ enligt $\lambda = \frac{2\pi}{k}$.

\item Vid givet $x$: samma fas då $t \mapsto t + \frac{2 \pi}{\omega}$. 

\item Våghastigheten finner vi genom att notera att $x-\frac{\omega}{k}t = \mathrm{konstant}$ beskriver punkter med samma fas. Detta ger
\end{itemize}

\noindent
$$
\mbox{d}x - \frac{\omega}{k} \mbox{d}t = 0 \quad \Rightarrow \quad \frac{\mbox{d}x}{\mbox{d}t} = \frac{\omega}{k}
$$
Hastigheten är alltså $v = \omega / k = c$. Ljushastigheten!
\end{notice_mdfboxadmon} % title: Exempel: Vågekvationen i $D=1$



I rummet kan vi skriva lösningarna
$$
\vec{E} = \vec{E}_0 \exp\left( i (\vec{k} \cdot \vec{r} - \omega t ) \right) = \left\{ \mathrm{välj~} \vec{k} = k \hat{x} \right\} = \vec{E}_0 \exp\left( i (k x - \omega t) \right).
$$
Den fysikaliska lösningen är (antingen) real- eller imaginärdelen av ovanstående. Det betyder alltså att lösningen och tolkningen är analog med $D=1$-exemplet ovan.


\paragraph{Elektriska och magnetiska vågor.}
En motsvarande vågekvation för magnetfältet kan också härledas. Man finner därför
\begin{align}
\vec{E}(\vec{r},t) &= \vec{E}_0 \exp\left( i (\vec{k} \cdot \vec{r} - \omega t ) \right)  \nonumber \\
\vec{B}(\vec{r},t) &= \vec{B}_0 \exp\left( i (\vec{k} \cdot \vec{r} - \omega t ) \right)  \nonumber 
\end{align}
Hur förhåller sig \emph{polarisationsvektorn} $\vec{E}_0$ till $\vec{B}_0$ och till riktningen på $\vec{k}$?
\begin{itemize}
\item Sätt $\vec{k} = \hat{n} k = \hat{n} \frac{\omega}{c}$.

\item Exponenten blir då $(\vec{k} \cdot \vec{r} - \omega t ) = \frac{\omega}{c}(\hat{n} \cdot \vec{r} - c t )$.

\item ME1: $\vnabla \cdot \vec{E} = 0$, vilket ger $\hat{n} \cdot \vec{E}_0 = 0$.
\end{itemize}

\noindent
Visa gärna detta med indexnotation
\begin{align}
\vnabla \cdot \vec{E} &= \partial_j E_{0,j} \exp \left( i \frac{\omega}{c} (n_l r_l - ct) \right) = E_{0,j} \left( \partial_j \left[i \frac{\omega}{c} n_l r_l \right] \right) \exp \left( i \frac{\omega}{c} (n_m r_m - ct) \right) \nonumber \\
&= \left\{ \partial_j r_l = \delta_{jl} \right\} = i \frac{\omega}{c} E_{0,j} n_j \exp \left( i \frac{\omega}{c} (n_m r_m - ct) \right) \nonumber \\
&= i \frac{\omega}{c} \hat{n} \cdot \vec{E}_0 \exp \left( i \frac{\omega}{c} (\hat{n} \cdot \vec{r} - ct) \right) \nonumber
\end{align}
\begin{itemize}
\item Pss ME3 ger $\hat{n} \cdot \vec{B}_0 = 0$

\item Fälten är alltså vinkelräta mot vågens rörelseriktning.

\item ME2 säger att $i \frac{\omega}{c} (\hat{n} \times \vec{E}_0 - c \vec{B}_0) = 0$, dvs $\vec{B}_0 = \frac{1}{c} \hat{n} \times \vec{E_0}$.

\item E- och B-fälten är alltså vinkelräta mot varandra. 

\item De två möjliga riktningarna på polarisationsvektorn $\vec{E}_0$ motsvarar de två möjliga polarisationerna hos elektromagnetisk strålning.
\end{itemize}

\noindent
Den elektromagnetiska vågen består därför av svängande elektriska och magnetiska fält, vilka genererar varandra
\begin{align}
\vec{E}(\vec{r},t) &= \vec{E}_0 \exp\left( i \frac{\omega}{c}(\hat{n} \cdot \vec{r} - c t ) \right), \quad \hat{n} \cdot \vec{E}_0 = 0  \\
\vec{B}(\vec{r},t) &= \frac{1}{c} \hat{n} \times \vec{E}_0 \exp\left( i \frac{\omega}{c}(\hat{n} \cdot \vec{r} - c t ) \right)  
\end{align}

\subsection*{Vågekvationer för potentialerna}

\begin{itemize}
\item Hur blir Maxwells ekvationer när fälten uttrycks i potentialerna $\phi$ och $\vec A$.

\item De två homogena ekvationerna ($\ldots = 0$) är de som gör att fälten kan uttryckas i termer av potentialer. 
\end{itemize}

\noindent
Sätt in uttrycken
\begin{align}
\vec{E} &= -\vnabla \phi - \frac{\partial \vec{A}}{\partial t} \nonumber \\
\vec{B} &= \vnabla \times \vec{A} \nonumber
\end{align}
i de inhomogena ekvationerna:
\begin{align*}
\Big( \vnabla \cdot \vec{E} &= \Big) \quad \vnabla\cdot(-\vnabla\phi-\frac{\partial \vec A}{\partial t}) = \frac{\rho}{\epsilon_0}, \\
\Big( \vnabla \times \vec{B} - \epsilon \mu_0 \frac{\partial \vec{E}}{\partial t}
&= \Big) \quad \vnabla\times(\vnabla\times\vec A) - \frac{1}{c^2} \frac{\partial}{\partial t} (-\vnabla\phi-\frac{\partial \vec A}{\partial t})=\mu_0\vec\jmath .
\end{align*}
\begin{itemize}
\item Ett gaugeval för vektorpotentialen förenklar ekvationerna avsevärt 

\item Välj gaugeparameter så att $\vnabla\cdot\vec A = - \frac{1}{c^2} \frac{\partial \phi}{\partial t}$. 
\end{itemize}

\noindent
Den första ekvationen blir
$$
\vnabla \cdot \left(-\vnabla \phi - \frac{\partial \vec{A}}{\partial t} \right) = -\Delta \phi - \frac{\partial}{\partial t} (\vnabla \cdot \vec{A}) = \left\{ \mathrm{Gaugeval} \right\} = \left( -\Delta + \frac{1}{c^2} \frac{\partial^2}{\partial t^2} \right) \phi \equiv - \square \phi,
$$
där d'Alembert-operatorn $\square \equiv \left( \Delta - \frac{1}{c^2} \frac{\partial^2}{\partial t^2} \right)$.

För den andra ekvationen utnyttjar vi att
$$
\vnabla \times \left( \vnabla \times \vec{A} \right) = \vnabla \left( \vnabla \cdot \vec{A} \right) - \Delta \vec{A} = \left\{ \mathrm{Gaugeval} \right\} = -\vnabla \left( \frac{1}{c^2} \frac{\partial \phi}{\partial t} \right) - \Delta \vec{A}.
$$
VL i den andra inhomogena ekvationen blir därför
$$ 
-\frac{1}{c^2} \frac{\partial (\vnabla \phi)}{\partial t} - \Delta \vec{A} + \frac{1}{c^2}
\frac{\partial \vnabla \phi}{\partial t} + \frac{1}{c^2} \frac{\partial^2 \vec{A}}{\partial  t^2} = -\square \vec{A}.
$$

\begin{itemize}
\item Vi får alltså inhomogena vågekvationer för potentialerna
\end{itemize}

\noindent
\begin{align*}
\square\phi&=-\frac{\rho}{\epsilon_0}, \\
\square\vec A&=-\mu_0\vec\jmath ,
\end{align*}
\begin{itemize}
\item Man kan använda Greensfunktionen för vågekvationen för att skriva ned generella lösningar när laddningar och strömmar är givna.

\item Greensfunktionen är lösningen till följande vågekvation
\end{itemize}

\noindent
$$
\square G(\vec{r},t;\vec{r}{\;}',t') \equiv (\Delta - \frac{1}{c^2} \frac{\partial^2}{\partial t^2}) G(\vec{r},t;\vec{r}{\;}',t') = -\delta^3(\vec{r}-\vec{r}{\;}')\delta(t-t').
$$
För enkelhets skull kan vi ta $\vec{r}{\;}'=0$, $t'=0$.

Greensfunktionen är
$$
G_+(\vec{r},t;0,0) = \frac{1}{4\pi t} \delta(r-ct).
$$  
\begin{itemize}
\item Detta är vad som brukar kallas en \emph{retarderad} Greensfunktion. 

\item Termen kommer sig av att en källa bara påverkar fält vid efterföljande tider (ljuskonen $r=ct$).
\end{itemize}

\noindent
\longinlinecomment{Kommentar 5}{ Ekvationen är symmetrisk under $t \to -t$, vilket innebär att det också finns en \emph{avancerad} Greensfunktion $G_- \propto \delta(r+ct)$. }{ Ekvationen är symmetrisk under }


\begin{warning_mdfboxadmon}[Huygens princip]
Det är värt att notera att det intuitivt ``sunda'' antagandet att Greensfunktionen endast har stöd på ljuskonen faktiskt är mer subtilt än det verkar. Det går under namnet Huygens pricip. Det visar sig vara sant i ett udda antal rumsdimensioner, men falskt i ett jämnt antal. I ett jämnt antal rumsdimensioner har Greensfunktionen stöd \emph{inom} ljuskonen, inte endast \emph{på} den.
I två dimensioner har man t.ex.
$$
G_+^{(2)}(\vec{r},t;0,0)=
\left\{ 
\begin{array}{ll}
\frac{\sigma(t)}{2\pi\sqrt{(ct)^2-r^2}}, & r<ct, \\
0 & \mathrm{annars.}
\end{array}
\right.
$$
Tänk på en sten som släpps i en vattenyta. En punkt på ytan påverkas inte bara just då vågfronten passerar, utan även vid alla senare tidpunkter. Tvådimensionell musik är kanske problematisk...
\end{warning_mdfboxadmon} % title: Huygens princip



Man måste förstås kunna bekräfta att man får rätt uttryck för en statisk laddningsfördelning. Om $\rho$ är oberoende av tiden ger Greensfunktionsmetoden
\begin{align*}
\phi(\vec{r},t) &= 
\int_{\mathbb{R}^3}dV' \int_{-\infty}^\infty dt'
\frac{\rho(\vec{r}{\;}')}{\epsilon_0} G_+(\vec{r},t,\vec{r}{\;}',t') \\
&= \int_{\mathbb{R}^3}dV' \int_{-\infty}^\infty dt'
\frac{\rho(\vec{r}{\;}')}{4\pi\epsilon_0(t-t')} \delta(|\vec{r}-\vec{r}{\;}'|-c(t-t')) \\
&= \left\{ c(t-t') \equiv (x-x') \; \Rightarrow c \mathrm{~försvinner} \right\} =\int_{\mathbb{R}^3}dV'
\frac{\rho(\vec{r}{\;}')}{4\pi\epsilon_0|\vec{r}-\vec{r}{\;}'|},
\end{align*}
vilket är identiskt med vad man får med hjälp av Greensfunktionen för Poissons ekvation.


% ------------------- end of main content ---------------

% #ifdef PREAMBLE
\end{document}
% #endif

