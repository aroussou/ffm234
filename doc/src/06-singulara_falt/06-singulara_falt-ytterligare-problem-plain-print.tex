%%
%% Automatically generated file from DocOnce source
%% (https://github.com/hplgit/doconce/)
%%
%%


%-------------------- begin preamble ----------------------

\documentclass[%
oneside,                 % oneside: electronic viewing, twoside: printing
final,                   % draft: marks overfull hboxes, figures with paths
10pt]{article}

\listfiles               % print all files needed to compile this document

\usepackage{relsize,makeidx,color,setspace,amsmath,amsfonts,amssymb}
\usepackage[table]{xcolor}
\usepackage{bm,microtype}

\usepackage[pdftex]{graphicx}

\usepackage[T1]{fontenc}
%\usepackage[latin1]{inputenc}
\usepackage{ucs}
\usepackage[utf8x]{inputenc}

\usepackage{lmodern}         % Latin Modern fonts derived from Computer Modern

% Hyperlinks in PDF:
\definecolor{linkcolor}{rgb}{0,0,0.4}
\usepackage{hyperref}
\hypersetup{
    breaklinks=true,
    colorlinks=true,
    linkcolor=linkcolor,
    urlcolor=linkcolor,
    citecolor=black,
    filecolor=black,
    %filecolor=blue,
    pdfmenubar=true,
    pdftoolbar=true,
    bookmarksdepth=3   % Uncomment (and tweak) for PDF bookmarks with more levels than the TOC
    }
%\hyperbaseurl{}   % hyperlinks are relative to this root

\setcounter{tocdepth}{2}  % number chapter, section, subsection

% prevent orhpans and widows
\clubpenalty = 10000
\widowpenalty = 10000

% --- end of standard preamble for documents ---


% insert custom LaTeX commands...

\raggedbottom
\makeindex

%-------------------- end preamble ----------------------

\begin{document}

% endif for #ifdef PREAMBLE


% ------------------- main content ----------------------



% ----------------- title -------------------------

\thispagestyle{empty}

\begin{center}
{\LARGE\bf
\begin{spacing}{1.25}
FFM232, Klassisk fysik och vektorfält - Föreläsningsanteckningar
\end{spacing}
}
\end{center}

% ----------------- author(s) -------------------------

\begin{center}
{\bf \href{{http://fy.chalmers.se/subatom/nt/}}{Christian Forssén}, Institutionen för fundamental fysik, Chalmers, Göteborg, Sverige${}^{}$} \\ [0mm]
\end{center}

\begin{center}
% List of all institutions:
\end{center}
    
% ----------------- end author(s) -------------------------

% --- begin date ---
\begin{center}
Sep 30, 2015
\end{center}
% --- end date ---

\vspace{1cm}


\subsection*{Ytterligare räkneproblem}

Här ges några exempel hur vi kan använda Gauss och Stokes satser på vektorfält med singulariteter. 

Här hanterar vi singulariteten genom att lägga in 
\begin{itemize}
\item en liten sfär med radien $\epsilon$ kring en punktkälla.

\item en cylinder kring $z$-axeln för en linjekälla.

\item en cirkel med radien $\epsilon$ runt $z$-axeln för en virveltråd.
\end{itemize}

\noindent
---------------------------------------------

\paragraph{Exempel: Punktkälla.}


\vspace{3mm}


Beräkna integralen
\begin{equation}
  \int_S \vec{F} \cdot \mbox{d}\vec{S} 
\end{equation}
där $S$ är ytan
\begin{equation}
  \rho = a - z, \,\,\,0 \le z \le a
\end{equation}
med en normal som pekar uppåt, och fältet ges av
\begin{equation}
  \vec{F} = F_0 \left[\left(\frac{a^2}{r^2} + \frac{r^2 \cos^3 \theta}{a^2}
\right) \hat{e}_{r} - \frac{r^2 \cos^2 \theta \sin \theta}{a^2} \hat{z}
\right].
\end{equation}
$F_0$ och $a$ är konstanter.

\paragraph{Lösning.}
Vi börjar med att studera ytan $S$ som är en kon med  spetsen i $z = a$.  Konen har sin öppna bottenyta vid $z = 0$. 

Om vi tittar på fältet, så ser vi genast att det har en punktkälla i origo.  För att tolka de övriga termerna så skriver vi om fältet som
\begin{equation}
  \vec{F} = F_0 \frac{a^2}{r^2}\hat{e}_{r} + F_0 \frac{r^2\cos^2 \theta}{a^2}
\left(\cos \theta \hat{e}_{r} - \sin \theta \hat{e}_{\theta}\right).
\end{equation}
Uttrycket inom parentesen är basvektorn $\hat{z}$, och $r\cos \theta = z$.  Vi kan därför dela upp fältet i två delar, en del som vi skriver i sfäriska koordinater, och som är singulär, och en andra del som vi skriver i kartesiska koordinater:
\begin{equation}
  \vec{F} = F_0 \left(\frac{a^2}{r^2} \hat{e}_{r} + \frac{z^2}{a^2} 
\hat{e}_{z}\right)
\end{equation}
Divergensen av $\vec{F}$ blir
\begin{equation}
  \nabla \cdot \vec{F} = \frac{2 F_0 z}{a^2}.
\end{equation}

För att använda Gauss sats behöver vi dels isolera singulariteten i origo genom att lägga in en halvsfär, $S_\epsilon$, med radien $\epsilon$ runt origo för $z >0$ och med en normalvektor som pekar mot origo.  
Dessutom måste vi sluta konen, vilket vi gör genom att lägga till en cirkelskiva, $S_1$,
med ytterradien $a$ och normalvektorn $-\hat{e}_{z}$ i $xy$-planet.

Gauss sats ger oss nu
\begin{equation}
  \int_S {\bf F\cdot \mbox{d}S} + \int_{S_\epsilon} \vec{F} \cdot \mbox{d}\vec{S}
+ \int_{S_1} \vec{F} \cdot \mbox{d}\vec{S} = \int_V {\bf \nabla \cdot F} \mbox{d}V.
\end{equation}
Vi börjar med att beräkna volymsintegralen
\begin{equation}
  \int_V \nabla \cdot \vec{F} \mbox{d}V = \int_V \frac{2\pi F_0 z}{a^2}
\mbox{d}V.
\end{equation}
Vi noterar här att vad vi gör är att integrera över cirkelskivor med radien $a -z$, så vår integral kan skrivas
\begin{equation}
  \int_0^a \frac{2 F_0 z}{a^2} \pi \left(a-z\right)^2 \mbox{d}z = 
\frac{2\pi F_0}{a^2}\left[\frac{z^4}{4} -\frac{2}{3} a z^3 + \frac{a^2 z^2}{2}
\right]_0^a = \frac{2\pi F_0}{a^2} \frac{a^4}{12} = \frac{\pi F_0 a^2}{6}.
\end{equation}
Sedan betraktar vi integralen över $S_\epsilon$ (lägg märke till att 
$r = \epsilon$ på $S_\epsilon$).
\begin{align}
  \int_{S_\epsilon} \vec{F} \cdot \mbox{d}\vec{S} &= \int_{S_\epsilon} 
F_0 \left(\frac{a^2}{\epsilon^2} \hat{e}_{r} + 
\frac{\epsilon^2 \cos^2 \theta}{a^2} \hat{e}_{z}\right) \cdot \left(-
\hat{e}_{r}\right) \mbox{d}S = - F_0 \left(\int_S \frac{a^2}{\epsilon^2}
\mbox{d}S + \int_S \frac{\epsilon^2 \cos^3 \theta}{a^2}\mbox{d}S\right) =
\nonumber \\
&- F_0 \left( \frac{a^2}{\epsilon^2} 2\pi \epsilon^2 + \frac{\epsilon^2}{a^2}
\int_S \cos^3 \theta \mbox{d}S\right) \to - 2\pi a^2 F_0 \,\,\,\,\mathrm{då }
\epsilon \to 0.
\end{align}
Slutligen har vi integralen över $S_1$
\begin{equation}
  \int_{S_1} \vec{F} \cdot \mbox{d}\vec{S} = \int_{S_1} F_0 \left(\frac{a^2}{r^2}
\hat{e}_{r} + \frac{z^2}{a^2} \hat{e}_{z}\right)\cdot\left(-\hat{e}_{z}
\right)\mbox{d}S.
\end{equation}
Tänk nu på att $\hat{e}_{r} \cdot \hat{z} = 0$ i $xy$-planet, och att $z = 0$ därstädes.  Därav följer att integralen är 0. Det följer nu att vår eftersökta integral blir
\begin{equation}
  \int_S \vec{F} \cdot \mbox{d}\vec{S} = \frac{\pi F_0 a^2}{6} + 2\pi F_0 a^2
= \frac{13\pi F_0 a^2}{6}.
\end{equation}

---------------------------------------------

\paragraph{Exempel: Linjekälla.}


\vspace{3mm}


Beräkna integralen
\begin{equation}
  \int_S \vec{F} \cdot \mbox{d}\vec{S},
\end{equation}
där ytan $S$ ges av $r = 2a$, $0 \le \varphi < 2\pi$ och $\frac{\pi}{4} \le \theta \le \frac{3\pi}{4}$ och har normalen $\hat{e}_{r}$, och fältet ges av
\begin{equation}
  \vec{F} = F_0 \left[\left(\frac{a}{\rho} + \frac{\rho}{a}\right) 
\hat{e}_{\rho} + \frac{z}{a} \hat{e}_{z}\right].
\end{equation}
$F_0$ och $a$ är här konstanter.

\paragraph{Lösning.}
Ytan $S$ är den mittersta delen av en sfär. Den avgränsas vid $z = \pm \sqrt{2} a$.

Fältet $\vec{F}$ är singulärt för $\rho = 0$, det vill säga längs $z$-axeln.  Fältets divergens är
\begin{equation}
  \nabla \cdot \vec{F} = F_0 \left[\frac{1}{\rho} \frac{\partial}{\partial \rho}
\left(a+ \frac{\rho^2}{a}\right) + \frac{\partial}{\partial z} \left(\frac{z}{a}
\right)\right] = F_0 \left(\frac{1}{\rho}\frac{2\rho}{a} + \frac{1}{a}\right) =
\frac{3 F_0}{a}.
\end{equation}

Vi sluter volymen genom att lägga till en cylinder, $S_\epsilon$ kring $z$-axeln mellan $-\sqrt{2}a \le z \le \sqrt{2}a$ och med radien $\epsilon$.  Normalen för $S_\epsilon$ är $-\hat{e}_{\rho}$.  Dessutom lägger vi till cirkulära skivor, $S_+$ och $S_-$, vid $z = \pm \sqrt{2}$ med normaler $\pm \hat{e}_{z}$.

Vi kan nu skriva Gauss sats som
\begin{equation}
\int_S \vec{F} \cdot \mbox{d}\vec{S} + \int_{S_\epsilon} \vec{F} \cdot \mbox{d}\vec{S} +
\int_{S_+} \vec{F} \cdot \mbox{d}\vec{S} + \int_{S_-} \vec{F} \cdot \mbox{d}\vec{S} =
\int_V \nabla \cdot \vec{F} \mbox{d}S.
\end{equation}
Vi börjar med volymsintegralen
\begin{equation}
  \int_V \nabla \cdot \vec{F} \mbox{d}S = \int_S \frac{3 F_0}{a} \mbox{d}V.
\end{equation}
Vi kan se den här integralen som om vi integrerar över cirkelskivor med radien $\sqrt{4 a^2 -z^2}$ för $-\sqrt{2}a \le z \le \sqrt{2}a$
\begin{align}
  \int_{-\sqrt{2}a}^{\sqrt{2}a} \frac{3F_0}{a} \pi \left(4a^2 - z^2\right)
\mbox{d}z &= \frac{3F_0}{a} 
\left[4a^2z - \frac{z^3}{3}\right]_{-\sqrt{2}a}^{\sqrt{2}a}
\nonumber \\
&= \frac{3F_0}{a} \left(4\sqrt{2}a^3 - \frac{2}{3}\sqrt{2}a^3 +
4\sqrt{2}a^3 - \frac{2}{3}\sqrt{2}a^3\right) \nonumber \\
&= 20 \sqrt{2} F_0 a^2.
\end{align}

Vi tar nu hand om ytintegralerna.  Först $S_\epsilon$, där $\rho = \epsilon$ på $S_\epsilon$
\begin{align}
  \int_{S_\epsilon} \vec{F} \cdot \mbox{d}\vec{S} &= \int_{S_\epsilon}  F_0 \left[
\left(\frac{a}{\epsilon} + \frac{\epsilon}{a}\right) \hat{e}_{\rho} + 
\frac{z}{a}\hat{e}_{z}\right) \cdot \left(-\hat{e}_{\rho}\right) \mbox{d}S \nonumber \\
& = \int_{S_\epsilon} F_0 \left(\frac{a}{\epsilon} + \frac{\epsilon}{a}\right)
\mbox{d}S = F_0 \left(\frac{a}{\epsilon} + \frac{\epsilon}{a}\right)
\int_{S_\epsilon} \mbox{d}S \nonumber \\
&= F_0 \left(\frac{a}{\epsilon} + \frac{\epsilon}{a}\right) 2\pi \epsilon 2\sqrt{2}
a \to 4\sqrt{2} \pi F_0 a^2 \,\,\,\,\mbox{då }\,\,\,\,\epsilon \to 0
\end{align}
På ytan $S_+$  har vi
\begin{equation}
  \int_{S_+} \vec{F} \cdot \mbox{d}\vec{S} = \int_{S_+} F_0 \left[\left(\frac{a}{\rho}
+ \frac{\rho}{a} \right) \hat{e}_{\rho} + \frac{z}{a}\hat{e}_{z}\right]
\cdot \hat{e}_{z} \mbox{d}S = \int_{S_+} F_0 \frac{z}{a} \mbox{d}S.
\end{equation}
På $S_+$ är $z = \sqrt{2}a$, vilket också är skivans ytterradie
\begin{equation}
  \int_{S_+} \vec{F} \cdot \mbox{d}\vec{S} = F_0 \sqrt{2} \pi 2a^2 = 2\sqrt{2} a^2F_0.\end{equation}
Vi inser att av symmetriskäl får integralen över $S_-$ samma värde.

Summerar vi nu ihop integralerna får vi
\begin{equation}
  \int_S \vec{F} \cdot \mbox{d}\vec{S} = 20 \sqrt{2} F_0 a^2 - 4 \sqrt{2} F_0 a^2
- 2 \times 2 \sqrt{2} F_0 a^2 = 12\sqrt{2} F_0 a^2.
\end{equation}

---------------------------------------------

\paragraph{Exempel: Virveltråd.}


\vspace{3mm}


Beräkna integralen
\begin{equation}
  \oint_C \vec{F} \cdot \mbox{d}\vec{r},
\end{equation}
där kurvan $C$ ges av 
\begin{equation}
  x^2 + \frac{y^2}{4} = a^2,\,\,\,\mbox{och}\,\,\,z = 0,
\end{equation}
som genomlöps i positiv riktning, och fältet ges av
\begin{equation}
  \vec{F} = F_0 \left[\frac{\rho \sin 2\varphi}{2a} \hat{e}_{\rho}
+ \left(\frac{a}{\rho} - \frac{\rho \sin^2 \varphi}{a}\right)\hat{e}_{\varphi}
\right].
\end{equation}
$F_0$ och $a$ är konstanter.

\paragraph{Lösning.}
Kurvan $C$ är en ellips med centrum i origo och halvaxlarna $a$ och $2a$. Enligt högerhandsregeln så väljer vi $\hat{e}_{z}$ som normalen till ellipsskivan.

Fältet $\vec{F}$ är singulärt på $z$-axeln.  Fältets rotation blir
\begin{equation}
  \nabla \times \vec{F} = \frac{F_0}{\rho} \left|
  \begin{array}{ccc}
\hat{e}_{\rho} & \rho \hat{e}_{\varphi} & \hat{e}_{z} \\
\frac{\partial}{\partial \rho} & \frac{\partial}{\partial \varphi} &
\frac{\partial}{\partial z} \\
\frac{\rho \sin 2\varphi}{2a} & a- 
\frac{\rho^2\sin 2 \varphi}{a} & 0\\
\end{array}
\right| = \frac{F_0}{\rho} \left(-\frac{2\rho \sin^2\varphi}{a} -
\frac{\rho\cos 2\varphi}{a}\right) \hat{e}_{z} =
- \frac{F_0}{a}\hat{e}_{z}.
\end{equation}

För att använda Stokes sats lägger in en cirkel, $C_\epsilon$ med radien $\epsilon$ runt origo.  Vi orienterar $C_\epsilon$ så att vi följer den medurs, det vill säga dess tangentvektor blir $-\hat{e}_{\varphi}$. Stokes sats ger oss sedan
\begin{equation}
  \oint_C \vec{F} \cdot \mbox{d}\vec{r} + \oint_{C_\epsilon} \vec{F} \cdot \mbox{d}\vec{r}
= \int_S {\bf \nabla \times F \cdot \mbox{d}S}.
\end{equation}
Först beräknar vi ytintegralen
\begin{equation}
  \int_S \nabla \times \vec{F} \cdot \mbox{d}\vec{S} = \int_S -\frac{F_0}{a} \hat{z} \cdot \hat{z} \mbox{d}S = - \frac{F_0}{a} \pi a\times 2a = -2\pi F_0 a,
\end{equation}
där vi har utnyttjat att ellipsens area är $2\pi a^2$.  Sedan beräknar vi integralen längs kurvan $C_\epsilon$ där $\rho = \epsilon$
\begin{align}
  \oint_{C_\epsilon} \vec{F} \cdot \mbox{d}\vec{r} &= \oint_{C_\epsilon} F_0 
\left[\frac{\rho \sin 2\varphi}{2a} \hat{e}_{\rho} + \left(\frac{a}{\rho}-
\frac{\rho \sin^2\varphi}{a}\right) \hat{e}_{\varphi}\right] \cdot 
\left(-\hat{e}_{\varphi}\right) \mbox{d}r \nonumber \\
&= -F_0\left( 
\oint_{C_\epsilon}\frac{a}{\epsilon} \mbox{d}r - \oint_{C_\epsilon}
\frac{\epsilon \sin^2 \varphi}{a} \mbox{d}r\right) \nonumber \\ 
&= -F_0 
\left(\frac{a}{\epsilon} 2\pi \epsilon - \frac{\epsilon}{a} \oint_{C_\epsilon}
\sin^2 \varphi \mbox{d}r\right) \to -2\pi F_0 a \,\,\,\,\mbox{då } \,\,\,\,
\epsilon \to 0.
\end{align}
Det följer att integralen blir
\begin{equation}
  \int_C \vec{F} \cdot \mbox{d}\vec{r} = 0.
\end{equation}

---------------------------------------------

% ------------------- end of main content ---------------


\printindex

\end{document}

