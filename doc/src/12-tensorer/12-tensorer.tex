%%
%% Automatically generated file from DocOnce source
%% (https://github.com/hplgit/doconce/)
%%

% #define PREAMBLE

% #ifdef PREAMBLE
%-------------------- begin preamble ----------------------

\documentclass[%
oneside,                 % oneside: electronic viewing, twoside: printing
final,                   % draft: marks overfull hboxes, figures with paths
10pt]{article}

\listfiles               %  print all files needed to compile this document

\usepackage{relsize,makeidx,color,setspace,amsmath,amsfonts,amssymb}
\usepackage[table]{xcolor}
\usepackage{bm,ltablex,microtype}

\usepackage[pdftex]{graphicx}

\usepackage[T1]{fontenc}
%\usepackage[latin1]{inputenc}
\usepackage{ucs}
\usepackage[utf8x]{inputenc}

\usepackage{lmodern}         % Latin Modern fonts derived from Computer Modern

% Hyperlinks in PDF:
\definecolor{linkcolor}{rgb}{0,0,0.4}
\usepackage{hyperref}
\hypersetup{
    breaklinks=true,
    colorlinks=true,
    linkcolor=linkcolor,
    urlcolor=linkcolor,
    citecolor=black,
    filecolor=black,
    %filecolor=blue,
    pdfmenubar=true,
    pdftoolbar=true,
    bookmarksdepth=3   % Uncomment (and tweak) for PDF bookmarks with more levels than the TOC
    }
%\hyperbaseurl{}   % hyperlinks are relative to this root

\setcounter{tocdepth}{2}  % levels in table of contents

\usepackage[framemethod=TikZ]{mdframed}

% --- begin definitions of admonition environments ---

% Admonition style "mdfbox" is an oval colored box based on mdframed
% "notice" admon
\definecolor{mdfbox_notice_background}{rgb}{1,1,1}
\newmdenv[
  skipabove=15pt,
  skipbelow=15pt,
  outerlinewidth=0,
  backgroundcolor=mdfbox_notice_background,
  linecolor=black,
  linewidth=2pt,       % frame thickness
  frametitlebackgroundcolor=mdfbox_notice_background,
  frametitlerule=true,
  frametitlefont=\normalfont\bfseries,
  shadow=false,        % frame shadow?
  shadowsize=11pt,
  leftmargin=0,
  rightmargin=0,
  roundcorner=5,
  needspace=0pt,
]{notice_mdfboxmdframed}

\newenvironment{notice_mdfboxadmon}[1][]{
\begin{notice_mdfboxmdframed}[frametitle=#1]
}
{
\end{notice_mdfboxmdframed}
}

% Admonition style "mdfbox" is an oval colored box based on mdframed
% "summary" admon
\definecolor{mdfbox_summary_background}{rgb}{1,1,1}
\newmdenv[
  skipabove=15pt,
  skipbelow=15pt,
  outerlinewidth=0,
  backgroundcolor=mdfbox_summary_background,
  linecolor=black,
  linewidth=2pt,       % frame thickness
  frametitlebackgroundcolor=mdfbox_summary_background,
  frametitlerule=true,
  frametitlefont=\normalfont\bfseries,
  shadow=false,        % frame shadow?
  shadowsize=11pt,
  leftmargin=0,
  rightmargin=0,
  roundcorner=5,
  needspace=0pt,
]{summary_mdfboxmdframed}

\newenvironment{summary_mdfboxadmon}[1][]{
\begin{summary_mdfboxmdframed}[frametitle=#1]
}
{
\end{summary_mdfboxmdframed}
}

% Admonition style "mdfbox" is an oval colored box based on mdframed
% "warning" admon
\definecolor{mdfbox_warning_background}{rgb}{1,1,1}
\newmdenv[
  skipabove=15pt,
  skipbelow=15pt,
  outerlinewidth=0,
  backgroundcolor=mdfbox_warning_background,
  linecolor=black,
  linewidth=2pt,       % frame thickness
  frametitlebackgroundcolor=mdfbox_warning_background,
  frametitlerule=true,
  frametitlefont=\normalfont\bfseries,
  shadow=false,        % frame shadow?
  shadowsize=11pt,
  leftmargin=0,
  rightmargin=0,
  roundcorner=5,
  needspace=0pt,
]{warning_mdfboxmdframed}

\newenvironment{warning_mdfboxadmon}[1][]{
\begin{warning_mdfboxmdframed}[frametitle=#1]
}
{
\end{warning_mdfboxmdframed}
}

% Admonition style "mdfbox" is an oval colored box based on mdframed
% "question" admon
\definecolor{mdfbox_question_background}{rgb}{1,1,1}
\newmdenv[
  skipabove=15pt,
  skipbelow=15pt,
  outerlinewidth=0,
  backgroundcolor=mdfbox_question_background,
  linecolor=black,
  linewidth=2pt,       % frame thickness
  frametitlebackgroundcolor=mdfbox_question_background,
  frametitlerule=true,
  frametitlefont=\normalfont\bfseries,
  shadow=false,        % frame shadow?
  shadowsize=11pt,
  leftmargin=0,
  rightmargin=0,
  roundcorner=5,
  needspace=0pt,
]{question_mdfboxmdframed}

\newenvironment{question_mdfboxadmon}[1][]{
\begin{question_mdfboxmdframed}[frametitle=#1]
}
{
\end{question_mdfboxmdframed}
}

% Admonition style "mdfbox" is an oval colored box based on mdframed
% "block" admon
\definecolor{mdfbox_block_background}{rgb}{1,1,1}
\newmdenv[
  skipabove=15pt,
  skipbelow=15pt,
  outerlinewidth=0,
  backgroundcolor=mdfbox_block_background,
  linecolor=black,
  linewidth=2pt,       % frame thickness
  frametitlebackgroundcolor=mdfbox_block_background,
  frametitlerule=true,
  frametitlefont=\normalfont\bfseries,
  shadow=false,        % frame shadow?
  shadowsize=11pt,
  leftmargin=0,
  rightmargin=0,
  roundcorner=5,
  needspace=0pt,
]{block_mdfboxmdframed}

\newenvironment{block_mdfboxadmon}[1][]{
\begin{block_mdfboxmdframed}[frametitle=#1]
}
{
\end{block_mdfboxmdframed}
}

% --- end of definitions of admonition environments ---

% prevent orhpans and widows
\clubpenalty = 10000
\widowpenalty = 10000

% --- end of standard preamble for documents ---


\usepackage[swedish]{babel}

\raggedbottom
\makeindex
\usepackage[totoc]{idxlayout}   % for index in the toc
\usepackage[nottoc]{tocbibind}  % for references/bibliography in the toc

%-------------------- end preamble ----------------------

\begin{document}

% matching end for #ifdef PREAMBLE
% #endif

\newcommand{\exercisesection}[1]{\subsection*{#1}}

% This file is to be run by preprocess to produce newcommands.tex
% to be included in .tex files.
% There are format-specific tests here for the newcommands (i.e.,
% different definitions of the commands depending on latex or mathjax).

% Newcommands for LaTeX math.
\newcommand{\tp}{\thinspace .}
\renewcommand{\Re}{\bbbr}
\newcommand{\Oof}[1]{\mathcal{O}(#1)}
\newcommand{\Prob}[1]{\hbox{P}(#1)}
\newcommand{\Var}[1]{\hbox{Var}(#1)}
\newcommand{\Cov}[2]{\hbox{Cov}(#1,#2)}
\newcommand{\StDev}[1]{\hbox{StDev}(#1)}

\newcommand{\punkt}{\thinspace .}
\newcommand{\komma}{\thinspace ,}

\newcommand{\vr}{\vec{r}}
\newcommand{\vrp}{\vec{r}\,'}
\newcommand{\erf}{\mathrm{erf}}
\newcommand{\vrho}{\vec{\varrho}}
\newcommand{\vrhop}{\vec{\varrho}\, '}
\newcommand{\sign}{\mathrm{sign}}

\newcommand{\Tr}[1]{\mathrm{Tr}[#1]}
\newcommand{\e}{\varepsilon}
\newcommand{\g}{\gamma}

\newcommand{\half}{\frac{1}{2}}
\newcommand{\vnabla}{\vec{\nabla}}


% Use footnotesize in subscripts
\newcommand{\subsc}[2]{#1_{\mbox{\footnotesize #2}}}




% ------------------- main content ----------------------



% ----------------- title -------------------------

\thispagestyle{empty}

\begin{center}
{\LARGE\bf
\begin{spacing}{1.25}
FFM234, Klassisk fysik och vektorfält - Föreläsningsanteckningar
\end{spacing}
}
\end{center}

% ----------------- author(s) -------------------------

\begin{center}
{\bf \href{{http://fy.chalmers.se/subatom/tsp/}}{Christian Forssén}, Institutionen för fysik, Chalmers, Göteborg, Sverige${}^{}$} \\ [0mm]
\end{center}

\begin{center}
% List of all institutions:
\end{center}
    
% ----------------- end author(s) -------------------------

% --- begin date ---
\begin{center}
Sep 16, 2019
\end{center}
% --- end date ---

\vspace{1cm}


\section*{12. Tensorer}

\begin{summary_mdfboxadmon}[Introduktion till tensorbegreppet]
\begin{itemize}
\item Fysikaliska lagar skall inte bero på i vilket koordinatsystem de beskrivs.

\item Detta kan vi åstadkomma genom att skriva dessa lagar som en likhet mellan två objekt vilka vi vet transformerar likadant under en koordinattransformation.

\item Vi kommer att kalla sådana objekt för \emph{tensorer}. 

\item T.ex. är en skalär en tensor av rank-0, en vektor är en tensor av rank-1.

\item En invarians hos sådana fysikaliska lagar kan kopplas till en symmetri (t.ex. rotationssymmetri). Invariansen kan kodas in genom att använda ``tensorspråket''.

\item Vi kommer att betrakta transformationer i rummet mellan Cartesiska system. Formalismen kan dock generaliseras till större symmetrier. T.ex. Lorentzinvarians i speciell relativitetsteori vilken involverar rum-tiden (fyra dimensioner); och mer allmänna koordinattransformationer i rum-tiden i allmän relativitetsteori.
\end{itemize}

\noindent
\end{summary_mdfboxadmon} % title: Introduktion till tensorbegreppet



Plan:
\begin{itemize}
\item Hur beskriva transformation mellan Cartesiska koordinatsystem

\item Transformationsegenskaper hos: Skalär, vektor, ... generell tensor

\item Visa att diverse objekt verkligen är tensorer

\item Fysikaliskt exempel: Tröghetstensorn
\end{itemize}

\noindent
\subsection*{Koordinattransformationer}

När vi byter koordinater från ett Cartesiskt högersystem, med koordinater
$x_i$, till ett annat (med origo i samma punkt), med koordinater $x'_i$, relateras 
\begin{equation}
x'_i=L_{ij}x_j,
\label{eq:transformation}
\end{equation}
där $\mathbf{L}$ är en ortogonal matris som uppfyller $\mathbf{L}\mathbf{L}^t = \mathbf{I}=\mathbf{L}^t\mathbf{L}$. Från detta följer direkt att $\det(\mathbf{L}^t\mathbf{L}) =  \det(\mathbf{L})^2 = \det\mathbf{I}=1$ vilket ger $\det\mathbf{L}=\pm 1$. 

Ett allmänt uttryck för determinanten av en $(3\times3)$-matris är 
\begin{equation}
\det\mathbf{M}=\varepsilon^{ijk}M_{1i}M_{2j}M_{3k}
\end{equation}

\begin{warning_mdfboxadmon}[Kommentar]
Kontrollera gärna. Detta uttryck ger en summa med sex nollskilda termer (3 positiva och 3 negativa) vilket alltså motsvarar determinanten.
\end{warning_mdfboxadmon} % title: Kommentar




\begin{warning_mdfboxadmon}[Rita]
$x'y'z'$-system som är roterat en vinkel $\alpha$ i $xy$-planet relativt ett $xyz$-system.
\begin{equation}
\begin{pmatrix}
x' \\ y' \\z'
\end{pmatrix}
= 
\begin{pmatrix}
\cos\alpha & \sin\alpha & 0\\
-\sin\alpha & \cos\alpha & 0 \\
0 & 0 & 1
\end{pmatrix}
\begin{pmatrix}
x \\ y \\ z
\end{pmatrix}
\end{equation}
\end{warning_mdfboxadmon} % title: Rita




\begin{warning_mdfboxadmon}[Kommentar]
Om $\mathbf{L}$ transformerar ett högersystem till ett högersystem gäller plustecknet, $\det\mathbf{L}=1$. Kolla exemplet ovan att $\mathbf{L}^t \mathbf{L} = \mathbf{I}$.
\end{warning_mdfboxadmon} % title: Kommentar



I indexnotation blir villkoret på transformationsmatrisen
\begin{equation}
\delta_{ij} = \left( L^t L \right)_{ij} = \left( L^t \right)_{ik} L_{kj} 
= L_{ki}L_{kj}.
\end{equation}

Notera att vi genom att derivera Ekv. (\ref{eq:transformation}) kan skriva matrisen $\mathbf{L}$ som $L_{ij}=\frac{\partial x'_i}{\partial x_j}$. Eftersom matrisen är ortogonal gäller den inversa relationen $x_i=L_{ji}x'_j$, och vi har därför också $L_{ji}=\frac{\partial x_i}{\partial x'_j}$.


\begin{notice_mdfboxadmon}[Bevis av invers relation]
Starta från transformationen $x'_i=L_{ij}x_j$ och multiplicera med inversa transformationen $\mathbf{L}^t$
\begin{equation}
(L^t)_{ki} x'_i=(L^t)_{ki} L_{ij}x_j = \delta_{kj} x_j = x_k,
\end{equation}
Vi noterar att $(L^t)_{ki} = L_{ik}$ i vänsterledet så att vi får den inversa relationen $L_{ik} x'_i = x_k$. Från detta uttryck byter vi bara namn på indexen: $i \mapsto j$, $k \mapsto i$ så att
\begin{equation}
x_i=L_{ji} x'_j.
\label{eq:inverstransformation}
\end{equation}
\end{notice_mdfboxadmon} % title: Bevis av invers relation



\subsection*{Skalärer och vektorer}
En skalär $s$ (= enklaste exempel på en tensor) kännetecknas av att den tar samma värde i alla koordinatsystem, dvs.~$s'=s$. 


\begin{warning_mdfboxadmon}[Kommentar]
Det är viktigt att förstå att det är transformationsregeln som är det viktiga. Det räcker inte med att ``$s$ är ett tal''.
\end{warning_mdfboxadmon} % title: Kommentar




\begin{warning_mdfboxadmon}[Rita]
En punkt P i ovanstående två koordinatsystem och illustrera dess $x'$- och $x$-koordinat.
\end{warning_mdfboxadmon} % title: Rita



Sålunda är $x$-koordinaten för en punkt i $\mathbf{R}^3$ inte en skalär, medan t.ex. temperaturen i en punkt är en skalär. 

En vektor $\vec v$ är en uppsättning tal som beter sig likadant som ortvektorns komponenter när vi byter system, dvs.
\begin{equation}
v'_i=L_{ij}v_j
\end{equation}

\begin{warning_mdfboxadmon}[Kommentar]
Det är denna transformationsregel som definierar vilka uppsättningar av tre (eller $D$) tal som får privilegiet att kallas vektor.
\end{warning_mdfboxadmon} % title: Kommentar



\subsection*{Tensorer}

\begin{warning_mdfboxadmon}[Kommentar]
Nu är det rättframt att gå vidare och definiera objekt,  tensorer, som har fler än ett index. En matris kan, som vi redan sett, skrivas som en tensor med två index, $T_{ij}$. Men inget hindrar att man har ett godtyckligt antal, säg $p$.
\end{warning_mdfboxadmon} % title: Kommentar



Transformationsregeln för en tensor med två index (tänk matris) är
\begin{equation} 
T'_{ij} = L_{ik} L_{jl} T_{kl} \left( = L_{ik} T_{kl} \left( L^t \right)_{lj} \right),
\end{equation}
vilket ju motsvarar $\mathbf{T}' = \mathbf{L} \mathbf{T} \mathbf{L}^t$.

En tensor med $p$ index (en tensor av rank $p$) skrivs $T_{i_1i_2\ldots i_p}$.  och allmänt
\begin{equation}
T'_{i_1\ldots i_p}=L_{i_1j_1}\ldots L_{i_pj_p}T_{j_1\ldots j_p}.
\end{equation}

Poängen med detta är att man kan multiplicera samman tensorer och
vara säker på att resultatet blir en tensor. 


\begin{notice_mdfboxadmon}[Skalärprodukt]
Är $\vec{u} \cdot \vec{v}$ en skalär? Hur beter den sig vid ett koordinatbyte?
\begin{equation}
u_i' v_i' = L_{ij} u_j L_{ik} v_k = \delta_{jk} u_j v_k = u_k v_k
\end{equation}
Resultatet är alltså oberoende av koordinatsystem, dvs det är en skalär. Ovanstående visar att summan av produkterna av två vektorers komponenter blir densamma när vektorerna uttrycks i det primmade respektive det oprimmade koordinatsystemet.
\end{notice_mdfboxadmon} % title: Skalärprodukt



\paragraph{Produkt av tensorer.}
Produkten av två tensorer är också en tensor.
\begin{itemize}
\item $c_{ij}=a_i b_j$ är också en tensor. 

\item $u_i=M_{ij}v_j$ är också en tensor. 
\end{itemize}

\noindent

\begin{warning_mdfboxadmon}[Kommentar]
För det andra fallet har vi $M'_{ij}v'_j=L_{ik}L_{jl}M_{kl}L_{jm}v_m=L_{ik}\delta_{lm}M_{kl}v_m=L_{ik}M_{kl}v_l
=L_{ik}u_k$ (där vi i första steget har använt att $\mathbf{L}$ är ortogonal). Om $M_{ij}$ och $v_i$ är tensorer är alltså även $u_i$ det. Det allmänna beviset går likadant (och innefattar förstås det faktum att skalärprodukten av två vektorer är en skalär).
\end{warning_mdfboxadmon} % title: Kommentar




\begin{notice_mdfboxadmon}[Exempel: Kryssprodukt]
Är $(\vec a\times\vec b)_i = \varepsilon_{ijk}a_jb_k$ en tensor? Dvs är resultatet en vektor? 


\begin{warning_mdfboxadmon}[Kommentar]
I linjär algebrakurserna har ni säkert visat att detta är en vektor.
\end{warning_mdfboxadmon} % title: Kommentar



Räknereglerna ovan ger att vi bara måste visa att $\varepsilon_{ijk}$ är en tensor. Transformationsreglerna säger:
\begin{equation}
\varepsilon'_{ijk}=L_{il}L_{jm}L_{kn}\varepsilon_{lmn}
\end{equation}
Vi kommer att visa att $\varepsilon$ är helt invariant under koordinattransformationer. Men låt oss först visa att $\varepsilon'_{ijk}$ är antisymmetriskt. Byt plats på två index
\begin{equation}
\varepsilon'_{jik}=L_{jl}L_{im}L_{kn}\varepsilon_{lmn} = L_{jm}L_{il}L_{kn}\varepsilon_{mln} = - L_{il} L_{jm}L_{kn}\varepsilon_{lmn} = - \varepsilon'_{ijk},
\end{equation}
där vi först bytte namn på två summationsindex, och sedan bytte plats på dem i $\varepsilon$-tensorn och utnyttjade att den är antisymmetrisk. 

Den bevisade antisymmetrin betyder ju också att element med två lika index måste vara noll. T.ex. $\varepsilon'_{iik} = -\varepsilon'_{iik}$.

$\varepsilon'_{ijk}$ är alltså proportionell mot $\varepsilon_{ijk}$. Visa därför en permutation, t.ex. $ijk=123$. 
\begin{equation}
\varepsilon'_{123}=L_{1l}L_{2m}L_{3n}\varepsilon_{lmn} = \det \mathbf{L} = +1,
\end{equation}
för en högertransformation.

Så $\varepsilon'_{ijk}=\varepsilon_{ijk}$, Levi--Civita-tensorn är en \emph{invariant tensor}. Den enda andra invarianta tensorn är Kroneckers delta. Visa själv att $\delta'_{ij}=\delta_{ij}$.
\end{notice_mdfboxadmon} % title: Exempel: Kryssprodukt



\paragraph{Vektoroperatorn.}
Vi behöver också kunna derivera. Låt oss visa det litet triviala påståendet att gradienten
av en skalär är en vektor. Vi utgår från skalärfältet $\phi$ och har alltså att $\phi' = \phi$. Vi har $(\vnabla\phi)'_i=\frac{\partial}{\partial x'_i}\phi'=\frac{\partial \phi}{\partial x_j} \frac{\partial x_j}{\partial x'_i}=
L_{ij} \frac{\partial \phi}{\partial x_j}=L_{ij}(\vnabla\phi)_j$. Detta visar att $\partial _i$ är en vektoroperator, och man kan sedan använda den för att konstruera andra derivator (divergens, rotation osv.). Samma regler gäller för $\partial _i$ som för andra tensorer.


\begin{notice_mdfboxadmon}[Tröghetstensorn]
Ett annat klassiskt exempel är tröghetstensorn,  
\begin{equation}
I_{ij}=\int_VdV\,\rho(r^2\delta_{ij}-x_ix_j),
\end{equation}
som man räknade ut i stelkroppsdynamik. Den relaterar rörelsemängdsmomentet till
rotationsvektorn enligt $L_i=I_{ij}\omega_j$. I och med att den innehåller upprepade
kryssprodukter är det enklare att härleda den i tensorformalism. 

Ett litet volymelement $dV$ av en stel kropp har massan $dm=\rho dV$,
och om kroppen roterar med en rotationsvektor $\omega_i$ har
det hastigheten $v_i=(\vec\omega\times\vec{r})_i=\varepsilon_{ijk}\omega_jx_k$. 
Dess rörelsemängd är $\mbox{d}p_i=\mbox{d}mv_i=\mbox{d}m\varepsilon_{ijk}\omega_jx_k$. Bidraget till rörelsemängdsmomentet från volymelementet är 
\begin{align}
\mbox{d}L_i = \varepsilon_{ijk}x_j \mbox{d}p_k &= \mbox{d}m\varepsilon_{ijk}x_j\varepsilon_{klm}\omega_lx_m
=\mbox{d}m(\delta_{il}\delta_{jm}-\delta_{im}\delta_{jl})x_jx_m\omega_l \nonumber \\
&=\mbox{d}m(r^2\omega_i-x_ix_j\omega_j)=\mbox{d}V\rho(r^2\delta_{ij}-x_ix_j)\omega_j
\end{align}
och totalt blir detta
\begin{equation}
L_i = \int_V \mbox{d}V\,\rho(r^2\delta_{ij}-x_ix_j) \omega_j \equiv I_{ij} \omega_j,
\end{equation}
där vi alltså definierar tröghetstensorn i det sista steget.
\end{notice_mdfboxadmon} % title: Tröghetstensorn




% ------------------- end of main content ---------------

% #ifdef PREAMBLE
\end{document}
% #endif

